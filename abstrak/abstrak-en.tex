\begin{center}
  \large\textbf{ABSTRACT}
  \label{chap:ABSTRACT}
\end{center}

\addcontentsline{toc}{chapter}{ABSTRACT}

\vspace{2ex}

\begingroup
% Menghilangkan padding
\setlength{\tabcolsep}{0pt}

\noindent
\begin{tabularx}{\textwidth}{l >{\centering}m{3em} X}
  \emph{Name}     & : & \name{}         \\

  \emph{Title}    & : & \engtatitle{}   \\

  \emph{Advisors} & : & 1. \advisor{}   \\
                  &   & 2. \coadvisor{} \\
\end{tabularx}
\endgroup

% Ubah paragraf berikut dengan abstrak dari tugas akhir dalam Bahasa Inggris
\emph{Course scheduling is crucial for the implementation of lecture activities. 
A schedule is said to be good if the resulting schedule can be carried out not only by the lecturers who teach, but also by the students who take the course. 
The process of preparing course scheduling at the Computer Engineering Department, ITS, is currently still being carried out conventionally. 
This conventional scheduling process can take a long time from the meeting process to the finished schedule. 
Constraints on the availability of lecturers, the number of courses, the number of rooms and the number of students are a challenge in the scheduling process 
because they must be considered so that there are no conflicts in the scheduling results. The problems that exist in the scheduling process for this course 
can be minimized by using existing technology so that an optimal scheduling process is produced according to the specified limitations. 
One method that can be used to overcome scheduling problems is to utilize the Genetic Algorithm method. 
A genetic algorithm is a technique for finding the optimal solution to a problem that has many solutions. 
This technique will look for solutions from several existing solutions until the best solution is obtained according to predetermined criteria. 
These criteria are known as constraints, which will be used for the fitness calculation process. 
By using this method, a schedule can be generated that meets predetermined criteria within 17 minutes 29 seconds.}

% Ubah kata-kata berikut dengan kata kunci dari tugas akhir dalam Bahasa Inggris
\emph{Keywords}: \emph{Algorithm}, \emph{Genetic}, \emph{Scheduling}.

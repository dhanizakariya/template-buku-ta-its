\begin{center}
  \large\textbf{ABSTRAK}
  \label{chap:ABSTRAK}
\end{center}

\addcontentsline{toc}{chapter}{ABSTRAK}

\vspace{2ex}

\begingroup
% Menghilangkan padding
\setlength{\tabcolsep}{0pt}

\noindent
\begin{tabularx}{\textwidth}{l >{\centering}m{2em} X}
  Nama Mahasiswa    & : & \name{}         \\

  Judul Tugas Akhir & : & \tatitle{}      \\

  Pembimbing        & : & 1. \advisor{}   \\
                    &   & 2. \coadvisor{} \\
\end{tabularx}
\endgroup

% Ubah paragraf berikut dengan abstrak dari tugas akhir
Penjadwalan mata kuliah merupakan hal krusial bagi 
terselenggaranya kegiatan perkuliahan. Sebuah penjadwalan dikatakan baik jika 
jadwal yang dihasilkan dapat dilaksanakan tidak hanya bagi dosen yang mengajar, 
tetapi juga oleh para mahasiswa yang mengambil mata kuliah tersebut. 
Proses penyusunan penjadwalan mata kuliah di Departemen Teknik Komputer ITS saat ini masih 
dilakukan secara konvensional. Proses penjadwalan konvensional ini bisa memakan waktu yang 
lama dari proses rapat hingga jadwal selesai. Kendala ketersediaan dosen, jumlah mata kuliah, 
jumlah ruangan dan jumlah mahasiswa menjadi tantangan dalam proses penjadwalan karena harus 
dipertimbangkan agar tidak terjadi bentrok dalam hasil penjadwalan. Masalah-masalah yang ada 
dalam proses penjadwalan mata kuliah ini bisa diminimalisir dengan menggunakan teknologi 
yang ada sehingga dihasilkan proses penjadwalan yang optimal sesuai dengan batasan-batasan yang ditentukan.
Salah satu metode yang dapat digunakan dalam mengatasi masalah penjadwalan adalah dengan memanfaatkan 
metode Algoritma Genetika. Algoritma Genetika merupakan teknik untuk mencari penyelesaian optimal dari 
sebuah permasalahan yang memiliki banyak solusi. Teknik ini akan mencari penyelesaian dari beberapa 
solusi yang ada sampai diperoleh penyelesaian terbaik sesuai dengan kriteria yang telah ditentukan sebelumnya. 
Kriteria-kriteria ini biasa dikenal dengan constraint yang akan digunakan untuk proses perhitungan fitness. 
Dengan menggunakan metode ini, dapat dihasilkan jadwal yang memenuhi kriteria yang telah ditentukan dalam waktu 17 menit 29 detik.

% Ubah kata-kata berikut dengan kata kunci dari tugas akhir
Kata Kunci: Algoritma, Genetikam, Penjadwalan.

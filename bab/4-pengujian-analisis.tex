\chapter{HASIL DAN PEMBAHASAN}
\label{chap:hasilpembahasan}
% Ubah bagian-bagian berikut dengan isi dari pengujian dan analisis

Pada penelitian ini dipaparkan hasil pengujian serta analisis dari desain sistem dan implementasi. 
Pengujian dilakukan untuk mengetahui keberhasilan sistem yang telah dibuat serta untuk mengetahui durasi pembuatan jadwal secara otomatis sehingga dapat ditarik kesimpulan terhadap penelitian tugas akhir ini. 
Pada penelitian ini dilakukan 2 jenis pengujian. 
Pengujian pertama pada penelitian ini dilakukan dengan menambah kompleksitas penghitungan nilai fitness dari kromosom penjadwalan secara bertahap dan melakukan variasi pada probabilitas mutasi untuk masing-masing peningkatan constrain.
Terdapat 4 \textit{constraint} yang digunakan pada perhitungan nilai fitness dan terdapat 5 variasi nilai probabilitas mutasi yang digunakan.
Sedangkan pada pengujian kedua dilakukan penambahan jumah populasi secara bertahap.

\section{Pengujian Pengaruh \textit{Constraint} dan Probabilitas Mutasi}
Pada pengujian ini yang dilakukan adalah meningkatkan kompleksitas perhitungan nilai fitness dengan meningkatkan jumlah \textit{constraint} secara bertahap. 
Masing-masing peningkatan constrain dilakukan variasi nilai probabilitas mutasi dan dilakukan pengulangan sebanyak 5 kali pengulangan.
Dari 5 kali pengulangan tersebut, diambil rata-rata waktu penyelesaian untuk masing-masing variasi probabilitas mutasi.
Terdapat 4 \textit{constraint} dan 5 variasi nilai probabilitas mutasi. Besar populasi yang digunakan pada pengujian ini adalah 2 individu.
\subsection{Percobaan dengan 1 \textit{Constraint}}
\label{sec:pengujian 1}
  Pada percobaan ini dilakukan perhitungan nilai fitness menggunakan 1 \textit{constraint} yaitu satu mata kuliah hanya bisa diselenggarakan satu kali dalam satu pekan perkuliahan.
  Kemudian dilakukan variasi nilai probabilitas mutasi dimana masing-masing nilai tersebut dilakukan 5 kali perulangan. 
  Berdasarkan pengujian, diperoleh rata-rata waktu seperti pada Tabel~\ref{tab:1 constraint}
  
  \begin{longtable}[c]{|c|c|c|}
    \caption{Rata-rata waktu proses penjadwalan dengan 1 \textit{constraint}}
    \label{tab:1 constraint}\\
    \hline
    \rowcolor[HTML]{C0C0C0} 
    Probabilitas Mutasi & Rata-rata waktu & Rata-rata Generasi \\ \hline
    0.02                & 4m 4s           & 54802               \\ \hline
    0.04                & 1m 47s          & 25894               \\ \hline
    0.06                & 1m 21s          & 19087               \\ \hline
    0.08                & 2m 28s          & 34845               \\ \hline
    0.1                 & 4m 58s          & 64808               \\ \hline
    \end{longtable}
  
  Berdasarkan data pada Tabel~\ref{tab:1 constraint}, semakin besar nilai probabilitas mutasi, maka waktu penyelesaian program akan semakin lama, selain itu ketika nilai probabilitas mutasi mendekati 0, waktu penyelesaian program juga semakin lama.
  Contoh hasil keluaran program pada pengujian ini dapat dilihat pada Tabel~\ref{tab:1} dibawah ini:
  \begin{small}
  \begin{longtable}[c]{|>{\centering\arraybackslash}m{1.1cm}|>{\centering\arraybackslash}m{1.1cm}|>{\centering\arraybackslash}m{1.7cm}|>{\centering\arraybackslash}m{4.7cm}|>{\centering\arraybackslash}m{4.7cm}|}
    \caption{Hasil Penjadwalan Otomatis dengan 1 \textit{Constraint}}
    \label{tab:1}\\
    \hline
    \rowcolor[HTML]{C0C0C0} 
    Hari   & Waktu       & Ruang /\linebreak Kapasitas & Matkul / Peserta                                 & Dosen                                         \\ \hline
    \endfirsthead
    \hline
    \rowcolor[HTML]{C0C0C0} 
    Hari   & Waktu       & Ruang /\linebreak Kapasitas & Matkul / Peserta                                 & Dosen                                         \\ \hline
    \endhead
    Senin  & 07.30-10.00 & A235 / 40                   & Desain Dan Rekayasa Sistem B / 40                & Mochamad Hariadi, S.T., M.Sc.,Ph.D            \\ \hline
    Senin  & 07.30-10.00 & AJ401 / 25                  & Rangkaian Listrik A / 40                         & Ir. Hany Boedinugroho, M.T.                   \\ \hline
    Senin  & 07.30-10.00 & A108 / 80                   & Metode Numerik A / 40                            & Mochamad Hariadi, S.T.,M.Sc.,Ph.D             \\ \hline
    Senin  & 10.00-12.30 & Pasca 2 / 40                & Rangkaian Digital A / 24                         & Ahmad Zaini, S.T., M.Sc.                      \\ \hline
    Senin  & 13.30-16.00 & A234 / 36                   & Metode Numerik C / 59                            & Atar Fuady Babgei, S.T., M.Sc.                \\ \hline
    Senin  & 13.30-16.00 & B211 / 25                   & Desain Dan Rekayasa Sistem A / 49                & Dion Hayu Fandiantoro, S.T., M.Eng.           \\ \hline
    Selasa & 07.30-10.00 & A234 / 36                   & Metode Numerik B / 40                            & Prof. Dr. Ir. Mauridhi Hery Purnomo, M.Eng.   \\ \hline
    Selasa & 07.30-10.00 & B211 / 25                   & Sistem Tertanam A / 27                           & Susi Juniastuti, S.T.,M.Eng.                  \\ \hline
    Selasa & 07.30-10.00 & AJ401 / 25                  & Sistem Operasi A / 25                            & Arief Kurniawan, S.T., M.T.                   \\ \hline
    Selasa & 10.00-12.30 & A235 / 40                   & Aljabar Linear A / 5                             & Dr. I Ketut Eddy Purnama, S.T.,M.T.           \\ \hline
    Selasa & 10.00-12.30 & A235 / 40                   & Pengantar Robotika A / 33                        & Prof. Dr. Ir. Yoyon Kusnendar Suprapto, M.Sc. \\ \hline
    Selasa & 10.00-12.30 & AJ402 / 25                  & Pemrograman Lanjut B / 39                        & Prof. Dr. Ir. Mauridhi Hery Purnomo, M.Eng.   \\ \hline
    Selasa & 13.30-16.00 & AJ401 / 25                  & Matematika Diskrit A / 32                        & Dr. Eko Mulyanto Yuniarno, S.T.,M.T.          \\ \hline
    Selasa & 13.30-16.00 & A234 / 36                   & Pemrograman Sistem dan Jaringan A / 11           & Reza Fuad Rachmadi, S.T.,M.T.,Ph.D            \\ \hline
    Selasa & 13.30-16.00 & Pasca 2 / 40                & Visi Komputer A / 16                             & Eko Pramunanto, S.T., M.T.                    \\ \hline
    Rabu   & 07.30-10.00 & Pasca 2 / 40                & Sistem Mikroprosesor dan Mikrokontroller A / 26  & Eko Pramunanto, S.T., M.T.                    \\ \hline
    Rabu   & 07.30-10.00 & A235 / 40                   & Probabilitas dan Statistik A / 28                & Reza Fuad Rachmadi, S.T.,M.T.,Ph.D            \\ \hline
    Rabu   & 10.00-12.30 & Pasca 2 / 40                & Persamaan Differensial dan Deret A / 35          & Ir. Hany Boedinugroho, M.T.                   \\ \hline
    Rabu   & 10.00-12.30 & B211 / 25                   & Persepsi Robot A / 13                            & Atar Fuady Babgei, S.T., M.Sc.                \\ \hline
    Rabu   & 10.00-12.30 & A234 / 36                   & Arsitektur dan Organisasi Sistem Komputer B / 27 & Dr. Surya Sumpeno, S.T.,M.Sc.                 \\ \hline
    Rabu   & 13.30-16.00 & AJ401 / 25                  & Sistem Tertanam B / 24                           & Dr. Diah Puspito Wulandari, S.T.,M.Sc.        \\ \hline
    Rabu   & 13.30-16.00 & B211 / 25                   & Jaringan Komputer B / 40                         & Prof. Dr. Ir. Mauridhi Hery Purnomo, M.Eng.   \\ \hline
    Rabu   & 13.30-16.00 & Pasca 2 / 40                & Sistem Manajemen Basis Data A / 25               & Dr. Supeno Mardi Susiki Nugroho, S.T., M.T.   \\ \hline
    Rabu   & 13.30-16.00 & AJ402 / 25                  & Deep Learning untuk Multimedia A / 30            & Prof. Dr. Ir. Yoyon Kusnendar Suprapto, M.Sc. \\ \hline
    Kamis  & 07.30-10.00 & AJ402 / 25                  & Rangkaian Listrik B / 35                         & Mochamad Hariadi, S.T.,M.Sc.,Ph.D             \\ \hline
    Kamis  & 07.30-10.00 & A235 / 40                   & Matematika Diskrit B / 59                        & Dion Hayu Fandiantoro, S.T., M.Eng.           \\ \hline
    Kamis  & 07.30-10.00 & A234 / 36                   & Pemrograman Lanjut A / 39                        & Prof. Dr. Ir. Mauridhi Hery Purnomo, M.Eng.   \\ \hline
    Kamis  & 10.00-12.30 & AJ402 / 25                  & Desain Pemrograman Game P / 8                    & Prof. Dr. Ir. Mauridhi Hery Purnomo, M.Eng.   \\ \hline
    Kamis  & 10.00-12.30 & AJ402 / 25                  & Sistem Manajemen Basis Data B / 25               & Muhtadin, S.T., M.T.                          \\ \hline
    Kamis  & 13.30-16.00 & A108 / 80                   & Desain Pemrograman Game A / 9                    & Muhtadin, S.T., M.T.                          \\ \hline
    Kamis  & 13.30-16.00 & AJ401 / 25                  & Jaringan Komputer A / 41                         & Eko Pramunanto, S.T., M.T.                    \\ \hline
    Jumat  & 07.30-10.00 & A108 / 80                   & Pengolahan Sinyal Digital A / 26                 & Eko Pramunanto, S.T., M.T.                    \\ \hline
    Jumat  & 07.30-10.00 & A235 / 40                   & Pengolahan Sinyal Digital B / 26                 & Ir. Hany Boedinugroho, M.T.                   \\ \hline
    Jumat  & 07.30-10.00 & A234 / 36                   & Sistem Mikroprosesor dan Mikrokontroller B / 20  & Arief Kurniawan, S.T., M.T.                   \\ \hline
    Jumat  & 13.30-16.00 & Pasca 2 / 40                & Dasar Pemrograman P / 14                         & Atar Fuady Babgei, S.T., M.Sc.                \\ \hline
    Jumat  & 13.30-16.00 & A235 / 40                   & Arsitektur dan Organisasi Sistem Komputer A / 30 & Muhtadin, S.T., M.T.                          \\ \hline
    Jumat  & 13.30-16.00 & A235 / 40                   & Sekuriti Sistem Komputer A / 33                  & Muhtadin, S.T., M.T.                          \\ \hline
    Jumat  & 13.30-16.00 & AJ402 / 25                  & Sistem Operasi B / 26                            & Dr. I Ketut Eddy Purnama, S.T.,M.T.           \\ \hline
    \end{longtable}
  \end{small}
  Data jadwal yang dihasilkan pada Tabel~\ref{tab:1}, masih terdapat ketidaksesuaian berupa satu ruangan dipakai untuk 2 mata kuliah pada satu waktu, jumlah peserta melebihi kapasitas ruangan dan satu dosen mengajar 2 mata kuliah pada satu waktu.
  Oleh sebab itu jadwal yang dihasilkan dari pengujian ini tidak bisa dipakai.
  
\subsection{Percobaan dengan 2 \textit{Constraint}}
\label{sec:pengujian 2}
Pada percobaan ini dilakukan perhitungan perhitungan nilai fitness menggunakan 2 \textit{constraint} yaitu:
\begin{enumerate}[nolistsep]
  \item Satu mata kuliah hanya bisa diselenggarakan satu kali dalam satu pekan perkuliahan.
  \item Satu dosen hanya bisa mengajar satu kali dalam satu waktu.
\end{enumerate} 
Kemudian dilakukan variasi nilai probabilitas mutasi dimana masing-masing nilai tersebut dilakukan 5 kali perulangan. 
Berdasarkan pengujian, diperoleh rata-rata waktu seperti pada Tabel~\ref{tab:2 constraint}

\begin{longtable}[c]{|c|c|c|}
  \caption{Rata-rata waktu proses penjadwalan dengan 2 \textit{constraint}}
  \label{tab:2 constraint}\\
  \hline
  \rowcolor[HTML]{C0C0C0} 
  Probabilitas Mutasi & Rata-rata waktu & Rata-rata Generasi \\ \hline
  0.02                & 4m 30s          & 49493               \\ \hline
  0.04                & 2m 2s           & 27551              \\ \hline
  0.06                & 1m 51s          & 26042               \\ \hline
  0.08                & 2m 41s          & 28991               \\ \hline
  0.1                 & 2m 57s          & 30440               \\ \hline
  \end{longtable}

Berdasarkan data pada Tabel~\ref{tab:2 constraint}, semakin besar nilai probabilitas mutasi, maka waktu penyelesaian program akan semakin lama, selain itu ketika nilai probabilitas mutasi mendekati 0, waktu penyelesaian program juga semakin lama.
Contoh hasil keluaran program pada pengujian ini adalah sebagai berikut:
\begin{small}
\begin{longtable}[c]{|>{\centering\arraybackslash}m{1.1cm}|>{\centering\arraybackslash}m{1.1cm}|>{\centering\arraybackslash}m{1.7cm}|>{\centering\arraybackslash}m{4.7cm}|>{\centering\arraybackslash}m{4.7cm}|}
  \caption{Hasil Penjadwalan Otomatis dengan 2 \textit{Constraint}}
  \label{tab:2}\\
  \hline
  \rowcolor[HTML]{C0C0C0} 
  Hari   & Waktu       & Ruang /\linebreak Kapasitas & Matkul / Peserta                                 & Dosen                                         \\ \hline
  \endfirsthead
  \hline
  \rowcolor[HTML]{C0C0C0} 
  Hari   & Waktu       & Ruang /\linebreak Kapasitas & Matkul / Peserta                                 & Dosen                                         \\ \hline
  \endhead
  Senin  & 07.30-10.00 & A234 / 36    & Probabilitas dan Statistik A / 28                & Dr. Supeno Mardi Susiki Nugroho, S.T., M.T.   \\ \hline
  Senin  & 07.30-10.00 & B211 / 25    & Sistem Manajemen Basis Data A / 25               & Dr. Surya Sumpeno, S.T.,M.Sc.                 \\ \hline
  Senin  & 10.00-12.30 & AJ401 / 25   & Jaringan Komputer A / 41                         & Dion Hayu Fandiantoro, S.T., M.Eng.           \\ \hline
  Senin  & 10.00-12.30 & A235 / 40    & Arsitektur dan Organisasi Sistem Komputer A / 30 & Dr. Eko Mulyanto Yuniarno, S.T.,M.T.          \\ \hline
  Senin  & 10.00-12.30 & Pasca 2 / 40 & Sistem Mikroprosesor dan Mikrokontroller B / 20  & Dr. Surya Sumpeno, S.T.,M.Sc.                 \\ \hline
  Senin  & 13.30-16.00 & B211 / 25    & Jaringan Komputer B / 40                         & Arief Kurniawan, S.T., M.T.                   \\ \hline
  Senin  & 13.30-16.00 & AJ401 / 25   & Sistem Mikroprosesor dan Mikrokontroller A / 26  & Dion Hayu Fandiantoro, S.T., M.Eng.           \\ \hline
  Senin  & 13.30-16.00 & A108 / 80    & Matematika Diskrit B / 59                        & Reza Fuad Rachmadi, S.T.,M.T.,Ph.D            \\ \hline
  Senin  & 13.30-16.00 & AJ401 / 25   & Metode Numerik C / 59                            & Susi Juniastuti, S.T.,M.Eng.                  \\ \hline
  Selasa & 07.30-10.00 & Pasca 2 / 40 & Rangkaian Digital A / 24                         & Atar Fuady Babgei, S.T., M.Sc.                \\ \hline
  Selasa & 07.30-10.00 & A234 / 36    & Pengolahan Sinyal Digital B / 26                 & Arief Kurniawan, S.T., M.T.                   \\ \hline
  Selasa & 10.00-12.30 & AJ402 / 25   & Desain Dan Rekayasa Sistem B / 40                & Ir. Hany Boedinugroho, M.T.                   \\ \hline
  Selasa & 13.30-16.00 & A108 / 80    & Metode Numerik B / 40                            & Prof. Dr. Ir. Mauridhi Hery Purnomo, M.Eng.   \\ \hline
  Rabu   & 10.00-12.30 & A234 / 36    & Rangkaian Listrik B / 35                         & Arief Kurniawan, S.T., M.T.                   \\ \hline
  Rabu   & 10.00-12.30 & Pasca 2 / 40 & Matematika Diskrit A / 32                        & Dr. Supeno Mardi Susiki Nugroho, S.T., M.T.   \\ \hline
  Rabu   & 10.00-12.30 & B211 / 25    & Sistem Operasi A / 25                            & Prof. Dr. Ir. Mauridhi Hery Purnomo, M.Eng.   \\ \hline
  Rabu   & 10.00-12.30 & B211 / 25    & Sistem Manajemen Basis Data B / 25               & Atar Fuady Babgei, S.T., M.Sc.                \\ \hline
  Rabu   & 13.30-16.00 & AJ401 / 25   & Deep Learning untuk Multimedia A / 30            & Dr. Surya Sumpeno, S.T.,M.Sc.                 \\ \hline
  Rabu   & 13.30-16.00 & A234 / 36    & Persepsi Robot A / 13                            & Arief Kurniawan, S.T., M.T.                   \\ \hline
  Rabu   & 13.30-16.00 & B211 / 25    & Persamaan Differensial dan Deret A / 35          & Dr. Diah Puspito Wulandari, S.T.,M.Sc.        \\ \hline
  Kamis  & 07.30-10.00 & B211 / 25    & Dasar Pemrograman P / 14                         & Dr. I Ketut Eddy Purnama, S.T.,M.T.           \\ \hline
  Kamis  & 07.30-10.00 & Pasca 2 / 40 & Desain Dan Rekayasa Sistem A / 49                & Dr. Supeno Mardi Susiki Nugroho, S.T., M.T.   \\ \hline
  Kamis  & 07.30-10.00 & AJ401 / 25   & Pemrograman Lanjut A / 39                        & Prof. Dr. Ir. Yoyon Kusnendar Suprapto, M.Sc. \\ \hline
  Kamis  & 07.30-10.00 & Pasca 2 / 40 & Sistem Operasi B / 26                            & Dion Hayu Fandiantoro, S.T., M.Eng.           \\ \hline
  Kamis  & 07.30-10.00 & Pasca 2 / 40 & Pemrograman Lanjut B / 39                        & Dr. Eko Mulyanto Yuniarno, S.T.,M.T.          \\ \hline
  Kamis  & 07.30-10.00 & Pasca 2 / 40 & Sistem Tertanam B / 24                           & Ir. Hany Boedinugroho, M.T.                   \\ \hline
  Kamis  & 07.30-10.00 & A108 / 80    & Arsitektur dan Organisasi Sistem Komputer B / 27 & Reza Fuad Rachmadi, S.T.,M.T.,Ph.D            \\ \hline
  Kamis  & 10.00-12.30 & A235 / 40    & Aljabar Linear A / 5                             & Susi Juniastuti, S.T.,M.Eng.                  \\ \hline
  Kamis  & 10.00-12.30 & A108 / 80    & Rangkaian Listrik A / 40                         & Eko Pramunanto, S.T., M.T.                    \\ \hline
  Kamis  & 10.00-12.30 & B211 / 25    & Metode Numerik A / 40                            & Reza Fuad Rachmadi, S.T.,M.T.,Ph.D            \\ \hline
  Kamis  & 13.30-16.00 & Pasca 2 / 40 & Desain Pemrograman Game A / 9                    & Prof. Dr. Ir. Mauridhi Hery Purnomo, M.Eng.   \\ \hline
  Kamis  & 13.30-16.00 & A234 / 36    & Sekuriti Sistem Komputer A / 33                  & Dr. Supeno Mardi Susiki Nugroho, S.T., M.T.   \\ \hline
  Jumat  & 07.30-10.00 & Pasca 2 / 40 & Pengolahan Sinyal Digital A / 26                 & Mochamad Hariadi, S.T.,M.Sc.,Ph.D             \\ \hline
  Jumat  & 07.30-10.00 & A234 / 36    & Desain Pemrograman Game P / 8                    & Prof. Dr. Ir. Yoyon Kusnendar Suprapto, M.Sc. \\ \hline
  Jumat  & 07.30-10.00 & A235 / 40    & Visi Komputer A / 16                             & Eko Pramunanto, S.T., M.T.                    \\ \hline
  Jumat  & 13.30-16.00 & A235 / 40    & Sistem Tertanam A / 27                           & Dion Hayu Fandiantoro, S.T., M.Eng.           \\ \hline
  Jumat  & 13.30-16.00 & A108 / 80    & Pengantar Robotika A / 33                        & Eko Pramunanto, S.T., M.T.                    \\ \hline
  Jumat  & 13.30-16.00 & AJ402 / 25   & Pemrograman Sistem dan Jaringan A / 11           & Dr. Diah Puspito Wulandari, S.T.,M.Sc.        \\ \hline
  \end{longtable}
\end{small}
Jadwal yang dihasilkan pada Tabel~\ref{tab:2} masih memiliki ketidaksesuaian yaitu satu ruangan dipakai untuk 2 perkuliahan dalam satu waktu dan terdapat jumlah peserta perkuliahan yang melebihi kapasitas ruangan, sehingga jadwal yang dihasilkan pada pengujian ini tidak bisa dipakai untuk perkuliahan.  

\subsection{Percobaan dengan 3 \textit{Constraint}}
\label{sec:pengujian 3}
Pada percobaan ini dilakukan perhitungan perhitungan nilai fitness menggunakan 3 \textit{constraint} yaitu:
\begin{enumerate}[nolistsep]
  \item Satu mata kuliah hanya bisa diselenggarakan satu kali dalam satu pekan perkuliahan.
  \item Satu dosen hanya bisa mengajar satu kali perkuliahan dalam satu waktu.
  \item Satu ruangan hanya bisa dipakai satu kali perkuliahan dalam satu waktu.
\end{enumerate} 
Kemudian dilakukan variasi nilai probabilitas mutasi dimana masing-masing nilai tersebut dilakukan 5 kali perulangan. 
Berdasarkan pengujian, diperoleh rata-rata waktu seperti pada Tabel~\ref{tab:3 constraint}

\begin{longtable}[c]{|c|c|c|}
  \caption{Rata-rata waktu proses penjadwalan dengan 3 \textit{constraint}}
  \label{tab:3 constraint}\\
  \hline
  \rowcolor[HTML]{C0C0C0} 
  Probabilitas Mutasi & Rata-rata waktu & Rata-rata Generasi\\ \hline
  0.02                & 10m 23s         & 104453            \\ \hline
  0.04                & 5m 27s          & 47884             \\ \hline
  0.06                & 4m 41s          & 41602             \\ \hline
  0.08                & 7m 45s          & 62040             \\ \hline
  0.1                 & 9m 3s           & 89413             \\ \hline
\end{longtable}

Berdasarkan data pada Tabel~\ref{tab:3 constraint}, semakin besar nilai probabilitas mutasi, maka waktu penyelesaian program akan semakin lama, selain itu ketika nilai probabilitas mutasi mendekati 0, waktu penyelesaian program juga semakin lama.
Contoh hasil keluaran program pada pengujian ini adalah sebagai berikut:
\begin{small}
  \begin{longtable}[c]{|>{\centering\arraybackslash}m{1.1cm}|>{\centering\arraybackslash}m{1.1cm}|>{\centering\arraybackslash}m{1.7cm}|>{\centering\arraybackslash}m{4.7cm}|>{\centering\arraybackslash}m{4.7cm}|}
  \caption{Hasil Penjadwalan Otomatis dengan 3 \textit{constraint}}
  \label{tab:3}\\
  \hline
  \rowcolor[HTML]{C0C0C0} 
  Hari   & Waktu       & Ruang /\linebreak Kapasitas & Matkul / Peserta                                 & Dosen                                         \\ \hline
  \endfirsthead
  \hline
  \rowcolor[HTML]{C0C0C0} 
  Hari   & Waktu       & Ruang /\linebreak Kapasitas & Matkul / Peserta                                 & Dosen                                         \\ \hline
  \endhead
    Senin  & 07.30-10.00 & AJ402 / 25   & Rangkaian Listrik A / 40                         & Ir. Hany Boedinugroho, M.T.                   \\ \hline
    Senin  & 07.30-10.00 & AJ401 / 25   & Dasar Pemrograman P / 14                         & Dr. Surya Sumpeno, S.T.,M.Sc.                 \\ \hline
    Senin  & 07.30-10.00 & A108 / 80    & Matematika Diskrit A / 32                        & Prof. Dr. Ir. Mauridhi Hery Purnomo, M.Eng.   \\ \hline
    Senin  & 10.00-12.30 & B211 / 25    & Pemrograman Sistem dan Jaringan A / 11           & Reza Fuad Rachmadi, S.T.,M.T.,Ph.D            \\ \hline
    Senin  & 13.30-16.00 & Pasca 2 / 40 & Jaringan Komputer B / 40                         & Dr. Supeno Mardi Susiki Nugroho, S.T., M.T.   \\ \hline
    Senin  & 13.30-16.00 & AJ401 / 25   & Desain Pemrograman Game A / 9                    & Ir. Hany Boedinugroho, M.T.                   \\ \hline
    Senin  & 13.30-16.00 & A235 / 40    & Jaringan Komputer A / 41                         & Dion Hayu Fandiantoro, S.T., M.Eng.           \\ \hline
    Selasa & 07.30-10.00 & AJ401 / 25   & Rangkaian Listrik B / 35                         & Eko Pramunanto, S.T., M.T.                    \\ \hline
    Selasa & 07.30-10.00 & B211 / 25    & Arsitektur dan Organisasi Sistem Komputer A / 30 & Dion Hayu Fandiantoro, S.T., M.Eng.           \\ \hline
    Selasa & 07.30-10.00 & A108 / 80    & Pengolahan Sinyal Digital A / 26                 & Dr. Diah Puspito Wulandari, S.T.,M.Sc.        \\ \hline
    Selasa & 10.00-12.30 & Pasca 2 / 40 & Sistem Tertanam B / 24                           & Muhtadin, S.T., M.T.                          \\ \hline
    Selasa & 10.00-12.30 & A234 / 36    & Aljabar Linear A / 5                             & Dr. Diah Puspito Wulandari, S.T.,M.Sc.        \\ \hline
    Selasa & 10.00-12.30 & A108 / 80    & Sistem Operasi A / 25                            & Susi Juniastuti, S.T.,M.Eng.                  \\ \hline
    Selasa & 10.00-12.30 & AJ402 / 25   & Desain Dan Rekayasa Sistem B / 40                & Mochamad Hariadi, S.T.,M.Sc.,Ph.D             \\ \hline
    Selasa & 13.30-16.00 & A234 / 36    & Pengantar Robotika A / 33                        & Dion Hayu Fandiantoro, S.T., M.Eng.           \\ \hline
    Rabu   & 07.30-10.00 & A108 / 80    & Desain Dan Rekayasa Sistem A / 49                & Atar Fuady Babgei, S.T., M.Sc.                \\ \hline
    Rabu   & 07.30-10.00 & Pasca 2 / 40 & Persamaan Differensial dan Deret A / 35          & Muhtadin, S.T., M.T.                          \\ \hline
    Rabu   & 07.30-10.00 & AJ401 / 25   & Sekuriti Sistem Komputer A / 33                  & Arief Kurniawan, S.T., M.T.                   \\ \hline
    Rabu   & 10.00-12.30 & Pasca 2 / 40 & Sistem Manajemen Basis Data A / 25               & Reza Fuad Rachmadi, S.T.,M.T.,Ph.D            \\ \hline
    Rabu   & 10.00-12.30 & A235 / 40    & Sistem Mikroprosesor dan Mikrokontroller A / 26  & Arief Kurniawan, S.T., M.T.                   \\ \hline
    Rabu   & 13.30-16.00 & A235 / 40    & Sistem Operasi B / 26                            & Mochamad Hariadi, S.T.,M.Sc.,Ph.D             \\ \hline
    Rabu   & 13.30-16.00 & A108 / 80    & Metode Numerik C / 59                            & Prof. Dr. Ir. Yoyon Kusnendar Suprapto, M.Sc. \\ \hline
    Rabu   & 13.30-16.00 & B211 / 25    & Deep Learning untuk Multimedia A / 30            & Eko Pramunanto, S.T., M.T.                    \\ \hline
    Kamis  & 07.30-10.00 & A234 / 36    & Matematika Diskrit B / 59                        & Dr. Eko Mulyanto Yuniarno, S.T.,M.T.          \\ \hline
    Kamis  & 07.30-10.00 & B211 / 25    & Arsitektur dan Organisasi Sistem Komputer B / 27 & Dr. Diah Puspito Wulandari, S.T.,M.Sc.        \\ \hline
    Kamis  & 07.30-10.00 & A235 / 40    & Sistem Mikroprosesor dan Mikrokontroller B / 20  & Atar Fuady Babgei, S.T., M.Sc.                \\ \hline
    Kamis  & 07.30-10.00 & A108 / 80    & Pemrograman Lanjut B / 39                        & Dr. Surya Sumpeno, S.T.,M.Sc.                 \\ \hline
    Kamis  & 07.30-10.00 & Pasca 2 / 40 & Metode Numerik A / 40                            & Reza Fuad Rachmadi, S.T.,M.T.,Ph.D            \\ \hline
    Kamis  & 07.30-10.00 & AJ402 / 25   & Pemrograman Lanjut A / 39                        & Susi Juniastuti, S.T.,M.Eng.                  \\ \hline
    Kamis  & 10.00-12.30 & B211 / 25    & Desain Pemrograman Game P / 8                    & Dr. I Ketut Eddy Purnama, S.T.,M.T.           \\ \hline
    Kamis  & 10.00-12.30 & A234 / 36    & Visi Komputer A / 16                             & Dr. Diah Puspito Wulandari, S.T.,M.Sc.        \\ \hline
    Kamis  & 10.00-12.30 & A235 / 40    & Sistem Manajemen Basis Data B / 25               & Dr. Surya Sumpeno, S.T.,M.Sc.                 \\ \hline
    Jumat  & 07.30-10.00 & A235 / 40    & Sistem Tertanam A / 27                           & Dr. Surya Sumpeno, S.T.,M.Sc.                 \\ \hline
    Jumat  & 07.30-10.00 & AJ401 / 25   & Pengolahan Sinyal Digital B / 26                 & Ahmad Zaini, S.T., M.Sc.                      \\ \hline
    Jumat  & 07.30-10.00 & A234 / 36    & Persepsi Robot A / 13                            & Mochamad Hariadi, S.T.,M.Sc.,Ph.D             \\ \hline
    Jumat  & 13.30-16.00 & AJ401 / 25   & Rangkaian Digital A / 24                         & Atar Fuady Babgei, S.T., M.Sc.                \\ \hline
    Jumat  & 13.30-16.00 & AJ402 / 25   & Probabilitas dan Statistik A / 28                & Reza Fuad Rachmadi, S.T.,M.T.,Ph.D            \\ \hline
    Jumat  & 13.30-16.00 & B211 / 25    & Metode Numerik B / 40                            & Ir. Hany Boedinugroho, M.T.                   \\ \hline
  \end{longtable}
\end{small}
Jadwal yang dihasilkan pada Tabel~\ref{tab:3} masih memiliki ketidaksesuaian karena terdapat jumlah peserta perkuliahan yang melebihi kapasitas ruangan sehingga jadwal yang dihasilkan pada pengujian ini tidak bisa dipakai untuk perkuliahan.

\subsection{Percobaan dengan 4 \textit{Constraint}}
\label{sec:pengujian 4}
Pada percobaan ini dilakukan perhitungan nilai fitness menggunakan 3 \textit{constraint} yaitu:
\begin{enumerate}[nolistsep]
  \item Satu mata kuliah hanya bisa diselenggarakan satu kali dalam satu pekan perkuliahan.
  \item Satu dosen hanya bisa mengajar satu kali perkuliahan dalam satu waktu.
  \item Satu ruangan hanya bisa dipakai satu kali perkuliahan dalam satu waktu.
  \item Jumlah peserta perkuliahan tidak boleh melebihi kapasitas ruangan
\end{enumerate} 
Kemudian dilakukan variasi nilai probabilitas mutasi dimana masing-masing nilai tersebut dilakukan 5 kali perulangan. 
Berdasarkan pengujian, diperoleh rata-rata waktu seperti pada Tabel~\ref{tab:3 constraint}

\begin{longtable}[c]{|c|c|c|}
  \caption{Rata-rata waktu proses penjadwalan dengan 4 \textit{constraint}}
  \label{tab:4 constraint}\\
  \hline
  \rowcolor[HTML]{C0C0C0} 
  Probabilitas Mutasi & Rata-rata waktu & Rata-rata Generasi \\ \hline
  0.02                & 1h 12m 29s      & 451003              \\ \hline
  0.04                & 38m 41s         & 302331              \\ \hline
  0.06                & 38m 27s         & 300065              \\ \hline
  0.08                & 38m 56s         & 311802              \\ \hline
  0.1                 & 57m 23s         & 405739              \\ \hline
  \end{longtable}

Berdasarkan data pada Tabel~\ref{tab:4 constraint}, semakin besar nilai probabilitas mutasi, maka waktu penyelesaian program akan semakin lama, selain itu ketika nilai probabilitas mutasi mendekati 0, waktu penyelesaian program juga semakin lama.
Berikut ini adalah contoh hasil keluaran program penjadwalan dengan konfigurasi 4 \textit{constraint}:
\begin{small}
\begin{longtable}[c]{|>{\centering\arraybackslash}m{1.1cm}|>{\centering\arraybackslash}m{1.1cm}|>{\centering\arraybackslash}m{1.7cm}|>{\centering\arraybackslash}m{4.7cm}|>{\centering\arraybackslash}m{4.7cm}|}
  \caption{Hasil Penjadwalan Otomatis dengan 4 \textit{constraint}}
  \label{tab:4}\\
  \hline
  \rowcolor[HTML]{C0C0C0} 
  Hari   & Waktu       & Ruang /\linebreak Kapasitas & Matkul / Peserta                                 & Dosen                                         \\ \hline
  \endfirsthead
  \hline
  \rowcolor[HTML]{C0C0C0} 
  Hari   & Waktu       & Ruang /\linebreak Kapasitas & Matkul / Peserta                                 & Dosen                                         \\ \hline
  \endhead
  Senin  & 07.30-10.00 & Pasca 2 / 40 & Visi Komputer A / 16                             & Arief Kurniawan, S.T., M.T.                   \\ \hline
  Senin  & 07.30-10.00 & A108 / 80    & Sistem Manajemen Basis Data A / 25               & Dr. Surya Sumpeno, S.T.,M.Sc.                 \\ \hline
  Senin  & 07.30-10.00 & A235 / 40    & Arsitektur dan Organisasi Sistem Komputer B / 27 & Atar Fuady Babgei, S.T., M.Sc.                \\ \hline
  Senin  & 10.00-12.30 & A235 / 40    & Jaringan Komputer B / 40                         & Prof. Dr. Ir. Yoyon Kusnendar Suprapto, M.Sc. \\ \hline
  Senin  & 10.00-12.30 & Pasca 2 / 40 & Matematika Diskrit A / 32                        & Dr. Supeno Mardi Susiki Nugroho, S.T., M.T.   \\ \hline
  Senin  & 13.30-16.00 & A235 / 40    & Rangkaian Listrik A / 40                         & Dr. Supeno Mardi Susiki Nugroho, S.T., M.T.   \\ \hline
  Senin  & 13.30-16.00 & Pasca 2 / 40 & Sistem Tertanam A / 27                           & Ir. Hany Boedinugroho, M.T.                   \\ \hline
  Selasa & 07.30-10.00 & B211 / 25    & Desain Pemrograman Game A / 9                    & Muhtadin, S.T., M.T.                          \\ \hline
  Selasa & 07.30-10.00 & A108 / 80    & Desain Dan Rekayasa Sistem A / 49                & Susi Juniastuti, S.T.,M.Eng.                  \\ \hline
  Selasa & 07.30-10.00 & Pasca 2 / 40 & Persamaan Differensial dan Deret A / 35          & Prof. Dr. Ir. Yoyon Kusnendar Suprapto, M.Sc. \\ \hline
  Selasa & 07.30-10.00 & A235 / 40    & Pengantar Robotika A / 33                        & Reza Fuad Rachmadi, S.T.,M.T.,Ph.D            \\ \hline
  Selasa & 10.00-12.30 & A234 / 36    & Sistem Mikroprosesor dan Mikrokontroller B / 20  & Dr. Supeno Mardi Susiki Nugroho, S.T., M.T.   \\ \hline
  Selasa & 10.00-12.30 & A108 / 80    & Jaringan Komputer A / 41                         & Dr. Eko Mulyanto Yuniarno, S.T.,M.T.          \\ \hline
  Selasa & 13.30-16.00 & A235 / 40    & Probabilitas dan Statistik A / 28                & Dr. Supeno Mardi Susiki Nugroho, S.T., M.T.   \\ \hline
  Selasa & 13.30-16.00 & A108 / 80    & Matematika Diskrit B / 59                        & Mochamad Hariadi, S.T.,M.Sc.,Ph.D             \\ \hline
  Selasa & 13.30-16.00 & Pasca 2 / 40 & Pengolahan Sinyal Digital B / 26                 & Susi Juniastuti, S.T.,M.Eng.                  \\ \hline
  Rabu   & 07.30-10.00 & A235 / 40    & Arsitektur dan Organisasi Sistem Komputer A / 30 & Dr. Diah Puspito Wulandari, S.T.,M.Sc.        \\ \hline
  Rabu   & 10.00-12.30 & A234 / 36    & Sistem Operasi B / 26                            & Ahmad Zaini, S.T., M.Sc.                      \\ \hline
  Rabu   & 10.00-12.30 & A108 / 80    & Rangkaian Listrik B / 35                         & Eko Pramunanto, S.T., M.T.                    \\ \hline
  Rabu   & 10.00-12.30 & A235 / 40    & Sistem Operasi A / 25                            & Arief Kurniawan, S.T., M.T.                   \\ \hline
  Rabu   & 13.30-16.00 & Pasca 2 / 40 & Pemrograman Lanjut A / 39                        & Susi Juniastuti, S.T.,M.Eng.                  \\ \hline
  Rabu   & 13.30-16.00 & A108 / 80    & Aljabar Linear A / 5                             & Reza Fuad Rachmadi, S.T.,M.T.,Ph.D            \\ \hline
  Rabu   & 13.30-16.00 & A234 / 36    & Deep Learning untuk Multimedia A / 30            & Dr. Eko Mulyanto Yuniarno, S.T.,M.T.          \\ \hline
  Kamis  & 07.30-10.00 & A108 / 80    & Sekuriti Sistem Komputer A / 33                  & Prof. Dr. Ir. Yoyon Kusnendar Suprapto, M.Sc. \\ \hline
  Kamis  & 07.30-10.00 & AJ402 / 25   & Sistem Manajemen Basis Data B / 25               & Arief Kurniawan, S.T., M.T.                   \\ \hline
  Kamis  & 07.30-10.00 & B211 / 25    & Sistem Operasi A / 25                            & Dr. Supeno Mardi Susiki Nugroho, S.T., M.T.   \\ \hline
  Kamis  & 07.30-10.00 & A234 / 36    & Pemrograman Sistem dan Jaringan A / 11           & Ir. Hany Boedinugroho, M.T.                   \\ \hline
  Kamis  & 07.30-10.00 & A235 / 40    & Pengolahan Sinyal Digital A / 26                 & Reza Fuad Rachmadi, S.T.,M.T.,Ph.D            \\ \hline
  Kamis  & 10.00-12.30 & A234 / 36    & Sistem Tertanam B / 24                           & Dr. Diah Puspito Wulandari, S.T.,M.Sc.        \\ \hline
  Kamis  & 10.00-12.30 & A235 / 40    & Pemrograman Lanjut B / 39                        & Dr. Supeno Mardi Susiki Nugroho, S.T., M.T.   \\ \hline
  Kamis  & 10.00-12.30 & AJ401 / 25   & Desain Dan Rekayasa Sistem B / 40                & Ir. Hany Boedinugroho, M.T.                   \\ \hline
  Kamis  & 13.30-16.00 & Pasca 2 / 40 & Sistem Mikroprosesor dan Mikrokontroller A / 26  & Dion Hayu Fandiantoro, S.T., M.Eng.           \\ \hline
  Kamis  & 13.30-16.00 & AJ402 / 25   & Dasar Pemrograman P / 14                         & Dr. Surya Sumpeno, S.T.,M.Sc.                 \\ \hline
  Kamis  & 13.30-16.00 & B211 / 25    & Desain Pemrograman Game P / 8                    & Susi Juniastuti, S.T.,M.Eng.                  \\ \hline
  Jumat  & 07.30-10.00 & Pasca 2 / 40 & Metode Numerik A / 40                            & Muhtadin, S.T., M.T.                          \\ \hline
  Jumat  & 07.30-10.00 & A108 / 80    & Persepsi Robot A / 13                            & Dion Hayu Fandiantoro, S.T., M.Eng.           \\ \hline
  Jumat  & 13.30-16.00 & AJ401 / 25   & Rangkaian Digital A / 24                         & Ahmad Zaini, S.T., M.Sc.                      \\ \hline
  Jumat  & 13.30-16.00 & A108 / 80    & Metode Numerik B / 40                            & Dr. Eko Mulyanto Yuniarno, S.T.,M.T.          \\ \hline
\end{longtable}
\end{small}

Jadwal pada Tabel~\ref{tab:4} telah memenuhi semua \textit{constraint} yang ada. 
Namun jadwal yang dihasilkan pada percobaan ini, semua dosen dianggap memiliki kompetensi yang sama.

% \section{Pengujian dengan Perubahan Data}
% \label{sec:Perubahan Data}
% Pada pengujian ini dilakukan perubahan data berupa penambahan jumlah ruangan

\section{Pengujian Pengaruh Besar Populasi}
\label{sec:populasi}
Pada pengujian ini yang dilakukan adalah meningkatkan jumlah individu pada populasi untuk melihat pengaruhnya terhadap kecepatan penyelesaian program dan jumlah generasi yang diperlukan.
Jumlah individu yang digunakan pada pengujian ini yaitu 2, 5, 10, 15, 20, 25, 30, 35, 40, 45, dan 50 populasi. 
Constraint yang digunakan pada pengujian ini hanya 1 yaitu satu mata kuliah hanya bisa dilaksanakan satu kali dalam satu pekan perkuliahan dan nilai probabilitas mutasi sebesar 0.06.
Berdasarkan pengujian yang dilakukan, diperoleh data seperti pada Tabel~\ref{tab:populasi} di bawah ini:

\begin{longtable}[c]{|c|c|c|}
  \caption{Pengaruh Besar Populasi Terhadap Durasi Penyelesaian Program}
  \label{tab:populasi}\\
  \hline
  \rowcolor[HTML]{9B9B9B} 
  Jumlah Individu & Waktu   & Generasi \\ \hline
  \endfirsthead
  \hline
  Jumlah Individu & Waktu   & Generasi \\ \hline
  \endhead
  %
  2               & 1m 42s  & 33411     \\ \hline
  5               & 2m 30s  & 33985    \\ \hline
  10              & 5m 12s  & 56943    \\ \hline
  15              & 10m 54s & 103969   \\ \hline
  20              & 9m      & 75289    \\ \hline
  25              & 6m 22s  & 48597    \\ \hline
  30              & 16m 4s  & 126662   \\ \hline
  35              & 2m 32s  & 21188    \\ \hline
  40              & 3m 42s  & 35224    \\ \hline
  45              & 1m 16s  & 11265    \\ \hline
  50              & 1m 32s  & 15782    \\ \hline
  \end{longtable}

Berdasarkan data yang diperoleh pada Tabel~\ref{tab:populasi}, jumlah individu pada populasi mempengaruhi durasi penyelesaian program. 
Penyelesaian tercepat diperoleh pada 45 individu, sedangkan yang terlama pada 30 individu.
\section{Pembahasan}
\label{sec:Pembahasan}

Dari pengujian yang telah dilakukan dapat diketahui bahwa semakin banyak batasan-batasan yang diberikan dalam proses algoritma genetika, maka kompleksitas dari proses \linebreak perhitungan evaluasi fitness juga akan semakin rumit. 
Kerumitan proses perhitungan ini tentu akan menambah durasi dari proses algoritma genetika itu sendiri. 
Hal ini dapat dilihat dari \linebreak peningkatan rata-rata waktu penyelesaian pada Tabel~\ref{tab:1}, Tabel~\ref{tab:2}, Tabel~\ref{tab:3}, dan Tabel~\ref{tab:4}.
Peningkatan waktu penyelesaian dari 1 \textit{constraint} ke 2 \textit{constraint} sebesar 34.4\%, dari 2 \linebreak \textit{constraint} ke 3 \textit{constraint} sebesar 166.071\%, dan dari 3 \textit{constraint} ke 4 \textit{constraint} sebesar 560.179\%.

Selain kompleksitas perhitungan evaluasi fitness dari satu individu, penentuan probabilitas mutasi juga sangat berpengaruh terhadap durasi proses algoritma genetika. 
Probabilitas mutasi dalam sebuah proses algoritma genetika tidak boleh terlalu besar, karena mengakibatkan hilangnya kemiripan antara individu baru dengan individu induknya. 
Sedangkan jika probabilitas mutasi bernilai terlalu kecil, maka durasi untuk memperoleh solusi berupa individu terbaik, akan memakan waktu yang sangat lama dan membuat proses algorima genetika dari penjadwalan otomatis ini menjadi tidak optimal.
Data yang diperoleh pada Tabel~\ref{tab:1 constraint}, Tabel~\ref{tab:2 constraint}, Tabel~\ref{tab:3 constraint}, dan Tabel~\ref{tab:4 constraint} menunjukkan bahwa proses algoritrma genetika dengan konfigurasi nilai probabilitas mutasi 0.06 memiliki rata-raata waktu penyelesaian program yang paling cepat pada semua variasi jumlah \textit{constraint} dengan rata-rata waktu 11m 35s, 
dan konfigurasi nilai probabilitas mutasi 0.02 memiliki rata-raata waktu penyelesaian program yang paling lama pada semua variasi jumlah \textit{constraint} dengan rata-rata waktu 22m 51s. 
Jadwal yang dihasilkan oleh program, terutama pada konfigurasi 4 \textit{constraint} (Tabel~\ref{tab:4}) menunjukkan bahwa program mampu menghasilkan jadwal yang sesuai dengan \textit{constraint} yang ditentukan. 
Jumlah peserta pada masing-masing kelas tidak ada yang melebihi kapasitas ruangan. 
Setiap mata kuliah juga hanya diselenggarakan satu kali dalam satu pekan perkuliahan.
Namun pada program ini, setiap dosen dianggap memiliki kompetensi yang sama dan hanya memperhatikan bahwa seorang dosen hanya bisa mengajar satu kali dalam satu waktu. 

Besar populasi juga mempengaruhi durasi proses algoritma genetika. 
Berdasarkan data pada Tabel~\ref{tab:populasi} durasi tercepat diperoleh pada 45 individu dengan waktu 1m 16ss dan 11265 generasi.
Sedangkan durasi terlama pada 35 individu dengan waktu 16m 4s dan 126662 generasi.
Dari data tersebut, dapat diketahui pula bahwa meskipun selisih durasi penyelesaian pada jumlah individu 2 dan 50 tidak terlalu besar.
Namun jumlah generasi pada 2 individu 2 kali lebih banyak daripada 50 individu. 
Hal ini dapat terjadi dikarenakan proses 1 kali iterasi pada 50 individu cenderung lebih lama daripada 1 kali iterasi pada 2 individu.
Hal ini menunjukkan bahwa semakin banyak jumlah individu dalam suatu populasi, menyebabkan rasio generasi dibandingkan waktu penyelesaian menjadi semakin kecil.
% Dari pengujian yang telah dilakukan dapat diketahui bahwa semakin banyak batasan-batasan yang diberikan dalam proses algoritma genetika, maka kompleksitas dari proses perhitungan evaluasi fitness juga akan semakin rumit. 
% Kerumitan proses perhitungan ini tentu akan menambah durasi dari proses algoritma genetika itu sendiri. Hal ini dapat dilihat dari perbedaan waktu antara Tabel~\ref{tab:tanpaKapasitas} dengan Tabel~\ref{tab:denganKapasitas}.

% Selain kompleksitas perhitungan evaluasi fitness dari satu individu, penentuan probabilitas mutasi juga sangat berpengaruh terhadap durasi proses algoritma genetika. Probabilitas \linebreak mutasi dalam sebuah proses algoritma genetika tidak boleh terlalu besar, karena mengakibatkan hilangnya kemiripan antara individu baru dengan individu induknya. 
% Sedangkan jika probabilitas mutasi bernilai terlalu kecil, maka durasi untuk memperoleh solusi berupa individu terbaik, akan memakan waktu yang sangat lama dan membuat proses algorima genetika dari penjadwalan otomatis ini menjadi tidak optimal. 

% Data yang diperoleh pada Tabel~\ref{tab:tanpaKapasitas} menunjukkan bahwa probabilitas mutasi 0.06 memiliki rata-rata durasi yang paling singkat yaitu dengan waktu 4 menit 54 detik. Sedangkan data pada Tabel~\ref{tab:denganKapasitas} menunjukkan bahwa probabilitas mutasi 0.05 memiliki rata-rata durasi yang paling singkat yakni dengan waktu 14 menit 53 detik. 
% % Contoh pembuatan tabel
% % \begin{longtable}{|c|c|c|}
% %   \caption{Hasil Pengukuran Energi dan Kecepatan}
% %   \label{tb:EnergiKecepatan}                                   \\
% %   \hline
% %   \rowcolor[HTML]{C0C0C0}
% %   \textbf{Energi} & \textbf{Jarak Tempuh} & \textbf{Kecepatan} \\
% %   \hline
% %   10 J            & 1000 M                & 200 M/s            \\
% %   20 J            & 2000 M                & 400 M/s            \\
% %   30 J            & 4000 M                & 800 M/s            \\
% %   40 J            & 8000 M                & 1600 M/s           \\
% %   \hline
% % \end{longtable}

% % \lipsum[2-4]

\chapter{HASIL DAN PEMBAHASAN}
\label{chap:hasilpembahasan}

% Ubah bagian-bagian berikut dengan isi dari pengujian dan analisis

Pada penelitian ini dipaparkan hasil pengujian serta analisis dari desain sistem dan implementasi. 
Pengujian dilakukan untuk mengetahui keberhasilan sistem yang telah dibuat serta untuk mengetahui durasi pembuatan jadwal secara otomatis sehingga dapat ditarik kesimpulan terhadap penelitian tugas akhir ini. 
Pengujian pada penelitian ini dilakukan dengan menambah kompleksitas penghitungan nilai fitness dari kromosom penjadwalan secara bertahap dan melakukan variasi pada probabilitas mutasi.
Terdapat 4 constraint yang digunakan pada perhitungan nilai fitness dan terdapat 5 variasi nilai probabilitas mutasi yang digunakan. 

\section{Percobaan dengan 1 constraint}
\label{sec:pengujian 1}
  Pada pengujian ini yang dilakukan adalah melakukan perhitungan nilai fitness menggunakan 1 constraint yaitu satu mata kuliah hanya bisa diselenggarakan satu kali dalam satu pekan perkuliahan.
  Kemudian dilakukan variasi nilai probabilitas mutasi dimana masing-masing nilai tersebut dilakukan 5 kali perulangan. 
  Berdasarkan pengujian, diperoleh rata-rata waktu seperti pada Tabel \ref{tab:1 constraint}
  \begin{longtable}[c]{|c|c|}
    \caption{Rata-rata waktu proses penjadwalan dengan 1 constraint}
    \label{tab:1 constraint}\\
    \hline
    \rowcolor[HTML]{C0C0C0} 
    Probabilitas Mutasi & Rata-rata waktu \\ \hline
    0.02                & 2m 45s          \\ \hline
    0.04                & 1m 54s          \\ \hline
    0.06                & 1m 53s          \\ \hline
    0.08                & 3m 8s           \\ \hline
    0.1                 & 4m 14s          \\ \hline
    \end{longtable}
  
  Berdasarkan data pada Tabel \ref{tab:1 constraint}, semakin besar nilai probabilitas mutasi, maka waktu penyelesaian program akan semakin lama, selain itu ketika nilai probabilitas mutasi mendekati 0, waktu penyelesaian program juga semakin lama.
  Contoh hasil keluaran program pada pengujian ini adalah sebagai berikut:
  \begin{longtable}[c]{|c|c|>{\centering\arraybackslash}m{2.5cm}|>{\centering\arraybackslash}m{4cm}|>{\centering\arraybackslash}m{3.2cm}|}
    \caption{Hasil Penjadwalan Otomatis dengan 1 Constraint}
    \label{tab:1}\\
    \hline
    \rowcolor[HTML]{C0C0C0} 
    Hari   & Waktu       & Ruang                  & Matkul                                                              & Dosen                                         \\ \hline
    Senin  & 07.30-10.00 & A235, Kapasitas: 40    & Desain Dan Rekayasa Sistem B, \linebreak Peserta: 40                & Mochamad Hariadi, S.T.,M.Sc.,Ph.D             \\ \hline
    Senin  & 07.30-10.00 & AJ401, Kapasitas: 25   & Rangkaian Listrik A, \linebreak Peserta: 40                         & Ir. Hany Boedinugroho, M.T.                   \\ \hline
    Senin  & 07.30-10.00 & A108, Kapasitas: 80    & Metode Numerik A, \linebreak Peserta: 40                            & Mochamad Hariadi, S.T.,M.Sc.,Ph.D             \\ \hline
    Senin  & 10.00-12.30 & Pasca 2, Kapasitas: 40 & Rangkaian Digital A, \linebreak Peserta: 24                         & Ahmad Zaini, S.T., M.Sc.                      \\ \hline
    Senin  & 13.30-16.00 & A234, Kapasitas: 36    & Metode Numerik C, \linebreak Peserta: 59                            & Atar Fuady Babgei, S.T., M.Sc.                \\ \hline
    Senin  & 13.30-16.00 & B211, Kapasitas: 25    & Desain Dan Rekayasa Sistem A, \linebreak Peserta: 49                & Dion Hayu Fandiantoro, S.T., M.Eng.           \\ \hline
    Selasa & 07.30-10.00 & A234, Kapasitas: 36    & Metode Numerik B, \linebreak Peserta: 40                            & Prof. Dr. Ir. Mauridhi Hery Purnomo, M.Eng.   \\ \hline
    Selasa & 07.30-10.00 & B211, Kapasitas: 25    & Sistem Tertanam A, \linebreak Peserta: 27                           & Susi Juniastuti, S.T.,M.Eng.                  \\ \hline
    Selasa & 07.30-10.00 & AJ401, Kapasitas: 25   & Sistem Operasi A, \linebreak Peserta: 25                            & Arief Kurniawan, S.T., M.T.                   \\ \hline
    Selasa & 10.00-12.30 & A235, Kapasitas: 40    & Aljabar Linear A, \linebreak Peserta: 5                             & Dr. I Ketut Eddy Purnama, S.T.,M.T.           \\ \hline
    Selasa & 10.00-12.30 & A235, Kapasitas: 40    & Pengantar Robotika A, \linebreak Peserta: 33                        & Prof. Dr. Ir. Yoyon Kusnendar Suprapto, M.Sc. \\ \hline
    Selasa & 10.00-12.30 & AJ402, Kapasitas: 25   & Pemrograman Lanjut B, \linebreak Peserta: 39                        & Prof. Dr. Ir. Mauridhi Hery Purnomo, M.Eng.   \\ \hline
    Selasa & 13.30-16.00 & AJ401, Kapasitas: 25   & Matematika Diskrit A, \linebreak Peserta: 32                        & Dr. Eko Mulyanto Yuniarno, S.T.,M.T.          \\ \hline
    Selasa & 13.30-16.00 & A234, Kapasitas: 36    & Pemrograman Sistem dan Jaringan A, \linebreak Peserta: 11           & Reza Fuad Rachmadi, S.T.,M.T.,Ph.D            \\ \hline
    Selasa & 13.30-16.00 & Pasca 2, Kapasitas: 40 & Visi Komputer A, \linebreak Peserta: 16                             & Eko Pramunanto, S.T., M.T.                    \\ \hline
    Rabu   & 07.30-10.00 & Pasca 2, Kapasitas: 40 & Sistem Mikroprosesor dan Mikrokontroller A, \linebreak Peserta: 26  & Eko Pramunanto, S.T., M.T.                    \\ \hline
    Rabu   & 07.30-10.00 & A235, Kapasitas: 40    & Probabilitas dan Statistik A, \linebreak Peserta: 28                & Reza Fuad Rachmadi, S.T.,M.T.,Ph.D            \\ \hline
    Rabu   & 10.00-12.30 & Pasca 2, Kapasitas: 40 & Persamaan Differensial dan Deret A, \linebreak Peserta: 35          & Ir. Hany Boedinugroho, M.T.                   \\ \hline
    Rabu   & 10.00-12.30 & B211, Kapasitas: 25    & Persepsi Robot A, \linebreak Peserta: 13                            & Atar Fuady Babgei, S.T., M.Sc.                \\ \hline
    Rabu   & 10.00-12.30 & A234, Kapasitas: 36    & Arsitektur dan Organisasi Sistem Komputer B, \linebreak Peserta: 27 & Dr. Surya Sumpeno, S.T.,M.Sc.                 \\ \hline
    Rabu   & 13.30-16.00 & AJ401, Kapasitas: 25   & Sistem Tertanam B, \linebreak Peserta: 24                           & Dr. Diah Puspito Wulandari, S.T.,M.Sc.        \\ \hline
    Rabu   & 13.30-16.00 & B211, Kapasitas: 25    & Jaringan Komputer B, \linebreak Peserta: 40                         & Prof. Dr. Ir. Mauridhi Hery Purnomo, M.Eng.   \\ \hline
    Rabu   & 13.30-16.00 & Pasca 2, Kapasitas: 40 & Sistem Manajemen Basis Data A, \linebreak Peserta: 25               & Dr. Supeno Mardi Susiki Nugroho, S.T., M.T.   \\ \hline
    Rabu   & 13.30-16.00 & AJ402, Kapasitas: 25   & Deep Learning untuk Multimedia A, \linebreak Peserta: 30            & Prof. Dr. Ir. Yoyon Kusnendar Suprapto, M.Sc. \\ \hline
    Kamis  & 07.30-10.00 & AJ402, Kapasitas: 25   & Rangkaian Listrik B, \linebreak Peserta: 35                         & Mochamad Hariadi, S.T.,M.Sc.,Ph.D             \\ \hline
    Kamis  & 07.30-10.00 & A235, Kapasitas: 40    & Matematika Diskrit B, \linebreak Peserta: 59                        & Dion Hayu Fandiantoro, S.T., M.Eng.           \\ \hline
    Kamis  & 07.30-10.00 & A234, Kapasitas: 36    & Pemrograman Lanjut A, \linebreak Peserta: 39                        & Prof. Dr. Ir. Mauridhi Hery Purnomo, M.Eng.   \\ \hline
    Kamis  & 10.00-12.30 & AJ402, Kapasitas: 25   & Desain Pemrograman Game P, \linebreak Peserta: 8                    & Prof. Dr. Ir. Mauridhi Hery Purnomo, M.Eng.   \\ \hline
    Kamis  & 10.00-12.30 & AJ402, Kapasitas: 25   & Sistem Manajemen Basis Data B, \linebreak Peserta: 25               & Muhtadin, S.T., M.T.                          \\ \hline
    Kamis  & 13.30-16.00 & A108, Kapasitas: 80    & Desain Pemrograman Game A, \linebreak Peserta: 9                    & Muhtadin, S.T., M.T.                          \\ \hline
    Kamis  & 13.30-16.00 & AJ401, Kapasitas: 25   & Jaringan Komputer A, \linebreak Peserta: 41                         & Eko Pramunanto, S.T., M.T.                    \\ \hline
    Jumat  & 07.30-10.00 & A108, Kapasitas: 80    & Pengolahan Sinyal Digital A, \linebreak Peserta: 26                 & Eko Pramunanto, S.T., M.T.                    \\ \hline
    Jumat  & 07.30-10.00 & A235, Kapasitas: 40    & Pengolahan Sinyal Digital B, \linebreak Peserta: 26                 & Ir. Hany Boedinugroho, M.T.                   \\ \hline
    Jumat  & 07.30-10.00 & A234, Kapasitas: 36    & Sistem Mikroprosesor dan Mikrokontroller B, \linebreak Peserta: 20  & Arief Kurniawan, S.T., M.T.                   \\ \hline
    Jumat  & 13.30-16.00 & Pasca 2, Kapasitas: 40 & Dasar Pemrograman P, \linebreak Peserta: 14                         & Atar Fuady Babgei, S.T., M.Sc.                \\ \hline
    Jumat  & 13.30-16.00 & A235, Kapasitas: 40    & Arsitektur dan Organisasi Sistem Komputer A, \linebreak Peserta: 30 & Muhtadin, S.T., M.T.                          \\ \hline
    Jumat  & 13.30-16.00 & A235, Kapasitas: 40    & Sekuriti Sistem Komputer A, \linebreak Peserta: 33                  & Muhtadin, S.T., M.T.                          \\ \hline
    Jumat  & 13.30-16.00 & AJ402, Kapasitas: 25   & Sistem Operasi B, \linebreak Peserta: 26                            & Dr. I Ketut Eddy Purnama, S.T.,M.T.           \\ \hline
    \end{longtable}
  Data jadwal yang dihasilkan pada Tabel \ref{tab:1}, masih terdapat ketidaksesuaian berupa satu ruangan dipakai untuk 2 mata kuliah pada satu waktu, jumlah peserta melebihi kapasitas ruangan dan satu dosen mengajar 2 mata kuliah pada satu waktu.
  Oleh sebab itu jadwal yang dihasilkan dari pengujian ini tidak bisa dipakai.
  
\section{Percobaan dengan 2 constraint}
\label{sec:pengujian 2}
Pada pengujian ini yang dilakukan adalah melakukan perhitungan nilai fitness menggunakan 2 constraint yaitu:
\begin{enumerate}[nolistsep]
  \item Satu mata kuliah hanya bisa diselenggarakan satu kali dalam satu pekan perkuliahan.
  \item Satu dosen hanya bisa mengajar satu kali dalam satu waktu.
\end{enumerate} 
Kemudian dilakukan variasi nilai probabilitas mutasi dimana masing-masing nilai tersebut dilakukan 5 kali perulangan. 
Berdasarkan pengujian, diperoleh rata-rata waktu seperti pada Tabel \ref{tab:2 constraint}
\begin{longtable}[c]{|c|c|}
  \caption{Rata-rata waktu proses penjadwalan dengan 2 constraint}
  \label{tab:2 constraint}\\
  \hline
  \rowcolor[HTML]{C0C0C0} 
  Probabilitas Mutasi & Rata-rata waktu \\ \hline
  0.02                & 5m 4s           \\ \hline
  0.04                & 1m 48s          \\ \hline
  0.06                & 1m 51s          \\ \hline
  0.08                & 2m 41s          \\ \hline
  0.1                 & 2m 57s          \\ \hline
  \end{longtable}

Berdasarkan data pada Tabel \ref{tab:2 constraint}, semakin besar nilai probabilitas mutasi, maka waktu penyelesaian program akan semakin lama, selain itu ketika nilai probabilitas mutasi mendekati 0, waktu penyelesaian program juga semakin lama.
Contoh hasil keluaran program pada pengujian ini adalah sebagai berikut:
\begin{longtable}[c]{|c|c|>{\centering\arraybackslash}m{2.5cm}|>{\centering\arraybackslash}m{4cm}|>{\centering\arraybackslash}m{3cm}|}
  \caption{Hasil Penjadwalan Otomatis dengan 2 Constraints}
  \label{tab:2}\\
  \hline
  \rowcolor[HTML]{C0C0C0} 
  Hari   & Waktu       & Ruang                  & Matkul                                                              & Dosen                                         \\ \hline
  Senin  & 07.30-10.00 & A234, Kapasitas: 36    & Probabilitas dan Statistik A, \linebreak Peserta: 28                & Dr. Supeno Mardi Susiki Nugroho, S.T., M.T.   \\ \hline
  Senin  & 07.30-10.00 & B211, Kapasitas: 25    & Sistem Manajemen Basis Data A, \linebreak Peserta: 25               & Dr. Surya Sumpeno, S.T.,M.Sc.                 \\ \hline
  Senin  & 10.00-12.30 & AJ401, Kapasitas: 25   & Jaringan Komputer A, \linebreak Peserta: 41                         & Dion Hayu Fandiantoro, S.T., M.Eng.           \\ \hline
  Senin  & 10.00-12.30 & A235, Kapasitas: 40    & Arsitektur dan Organisasi Sistem Komputer A, \linebreak Peserta: 30 & Dr. Eko Mulyanto Yuniarno, S.T.,M.T.          \\ \hline
  Senin  & 10.00-12.30 & Pasca 2, Kapasitas: 40 & Sistem Mikroprosesor dan Mikrokontroller B, \linebreak Peserta: 20  & Dr. Surya Sumpeno, S.T.,M.Sc.                 \\ \hline
  Senin  & 13.30-16.00 & B211, Kapasitas: 25    & Jaringan Komputer B, \linebreak Peserta: 40                         & Arief Kurniawan, S.T., M.T.                   \\ \hline
  Senin  & 13.30-16.00 & AJ401, Kapasitas: 25   & Sistem Mikroprosesor dan Mikrokontroller A, \linebreak Peserta: 26  & Dion Hayu Fandiantoro, S.T., M.Eng.           \\ \hline
  Senin  & 13.30-16.00 & A108, Kapasitas: 80    & Matematika Diskrit B, \linebreak Peserta: 59                        & Reza Fuad Rachmadi, S.T.,M.T.,Ph.D            \\ \hline
  Senin  & 13.30-16.00 & AJ401, Kapasitas: 25   & Metode Numerik C, \linebreak Peserta: 59                            & Susi Juniastuti, S.T.,M.Eng.                  \\ \hline
  Selasa & 07.30-10.00 & Pasca 2, Kapasitas: 40 & Rangkaian Digital A, \linebreak Peserta: 24                         & Atar Fuady Babgei, S.T., M.Sc.                \\ \hline
  Selasa & 07.30-10.00 & A234, Kapasitas: 36    & Pengolahan Sinyal Digital B, \linebreak Peserta: 26                 & Arief Kurniawan, S.T., M.T.                   \\ \hline
  Selasa & 10.00-12.30 & AJ402, Kapasitas: 25   & Desain Dan Rekayasa Sistem B, \linebreak Peserta: 40                & Ir. Hany Boedinugroho, M.T.                   \\ \hline
  Selasa & 13.30-16.00 & A108, Kapasitas: 80    & Metode Numerik B, \linebreak Peserta: 40                            & Prof. Dr. Ir. Mauridhi Hery Purnomo, M.Eng.   \\ \hline
  Rabu   & 10.00-12.30 & A234, Kapasitas: 36    & Rangkaian Listrik B, \linebreak Peserta: 35                         & Arief Kurniawan, S.T., M.T.                   \\ \hline
  Rabu   & 10.00-12.30 & Pasca 2, Kapasitas: 40 & Matematika Diskrit A, \linebreak Peserta: 32                        & Dr. Supeno Mardi Susiki Nugroho, S.T., M.T.   \\ \hline
  Rabu   & 10.00-12.30 & B211, Kapasitas: 25    & Sistem Operasi A, \linebreak Peserta: 25                            & Prof. Dr. Ir. Mauridhi Hery Purnomo, M.Eng.   \\ \hline
  Rabu   & 10.00-12.30 & B211, Kapasitas: 25    & Sistem Manajemen Basis Data B, \linebreak Peserta: 25               & Atar Fuady Babgei, S.T., M.Sc.                \\ \hline
  Rabu   & 13.30-16.00 & AJ401, Kapasitas: 25   & Deep Learning untuk Multimedia A, \linebreak Peserta: 30            & Dr. Surya Sumpeno, S.T.,M.Sc.                 \\ \hline
  Rabu   & 13.30-16.00 & A234, Kapasitas: 36    & Persepsi Robot A, \linebreak Peserta: 13                            & Arief Kurniawan, S.T., M.T.                   \\ \hline
  Rabu   & 13.30-16.00 & B211, Kapasitas: 25    & Persamaan Differensial dan Deret A, \linebreak Peserta: 35          & Dr. Diah Puspito Wulandari, S.T.,M.Sc.        \\ \hline
  Kamis  & 07.30-10.00 & B211, Kapasitas: 25    & Dasar Pemrograman P, \linebreak Peserta: 14                         & Dr. I Ketut Eddy Purnama, S.T.,M.T.           \\ \hline
  Kamis  & 07.30-10.00 & Pasca 2, Kapasitas: 40 & Desain Dan Rekayasa Sistem A, \linebreak Peserta: 49                & Dr. Supeno Mardi Susiki Nugroho, S.T., M.T.   \\ \hline
  Kamis  & 07.30-10.00 & AJ401, Kapasitas: 25   & Pemrograman Lanjut A, \linebreak Peserta: 39                        & Prof. Dr. Ir. Yoyon Kusnendar Suprapto, M.Sc. \\ \hline
  Kamis  & 07.30-10.00 & Pasca 2, Kapasitas: 40 & Sistem Operasi B, \linebreak Peserta: 26                            & Dion Hayu Fandiantoro, S.T., M.Eng.           \\ \hline
  Kamis  & 07.30-10.00 & Pasca 2, Kapasitas: 40 & Pemrograman Lanjut B, \linebreak Peserta: 39                        & Dr. Eko Mulyanto Yuniarno, S.T.,M.T.          \\ \hline
  Kamis  & 07.30-10.00 & Pasca 2, Kapasitas: 40 & Sistem Tertanam B, \linebreak Peserta: 24                           & Ir. Hany Boedinugroho, M.T.                   \\ \hline
  Kamis  & 07.30-10.00 & A108, Kapasitas: 80    & Arsitektur dan Organisasi Sistem Komputer B, \linebreak Peserta: 27 & Reza Fuad Rachmadi, S.T.,M.T.,Ph.D            \\ \hline
  Kamis  & 10.00-12.30 & A235, Kapasitas: 40    & Aljabar Linear A, \linebreak Peserta: 5                             & Susi Juniastuti, S.T.,M.Eng.                  \\ \hline
  Kamis  & 10.00-12.30 & A108, Kapasitas: 80    & Rangkaian Listrik A, \linebreak Peserta: 40                         & Eko Pramunanto, S.T., M.T.                    \\ \hline
  Kamis  & 10.00-12.30 & B211, Kapasitas: 25    & Metode Numerik A, \linebreak Peserta: 40                            & Reza Fuad Rachmadi, S.T.,M.T.,Ph.D            \\ \hline
  Kamis  & 13.30-16.00 & Pasca 2, Kapasitas: 40 & Desain Pemrograman Game A, \linebreak Peserta: 9                    & Prof. Dr. Ir. Mauridhi Hery Purnomo, M.Eng.   \\ \hline
  Kamis  & 13.30-16.00 & A234, Kapasitas: 36    & Sekuriti Sistem Komputer A, \linebreak Peserta: 33                  & Dr. Supeno Mardi Susiki Nugroho, S.T., M.T.   \\ \hline
  Jumat  & 07.30-10.00 & Pasca 2, Kapasitas: 40 & Pengolahan Sinyal Digital A, \linebreak Peserta: 26                 & Mochamad Hariadi, S.T.,M.Sc.,Ph.D             \\ \hline
  Jumat  & 07.30-10.00 & A234, Kapasitas: 36    & Desain Pemrograman Game P, \linebreak Peserta: 8                    & Prof. Dr. Ir. Yoyon Kusnendar Suprapto, M.Sc. \\ \hline
  Jumat  & 07.30-10.00 & A235, Kapasitas: 40    & Visi Komputer A, \linebreak Peserta: 16                             & Eko Pramunanto, S.T., M.T.                    \\ \hline
  Jumat  & 13.30-16.00 & A235, Kapasitas: 40    & Sistem Tertanam A, \linebreak Peserta: 27                           & Dion Hayu Fandiantoro, S.T., M.Eng.           \\ \hline
  Jumat  & 13.30-16.00 & A108, Kapasitas: 80    & Pengantar Robotika A, \linebreak Peserta: 33                        & Eko Pramunanto, S.T., M.T.                    \\ \hline
  Jumat  & 13.30-16.00 & AJ402, Kapasitas: 25   & Pemrograman Sistem dan Jaringan A, \linebreak Peserta: 11           & Dr. Diah Puspito Wulandari, S.T.,M.Sc.        \\ \hline
  \end{longtable}
Jadwal yang dihasilkan pada Tabel \ref{tab:2} masih memiliki ketidaksesuaian yaitu satu ruangan dipakai untuk 2 perkuliahan dalam satu waktu dan terdapat jumlah peserta perkuliahan yang melebihi kapasitas ruangan, sehingga jadwal yang dihasilkan pada pengujian ini tidak bisa dipakai untuk perkuliahan.  

\section{Percobaan dengan 3 constraint}
\label{sec:pengujian 3}
Pada pengujian ini yang dilakukan adalah melakukan perhitungan nilai fitness menggunakan 3 constraint yaitu:
\begin{enumerate}[nolistsep]
  \item Satu mata kuliah hanya bisa diselenggarakan satu kali dalam satu pekan perkuliahan.
  \item Satu dosen hanya bisa mengajar satu kali perkuliahan dalam satu waktu.
  \item Satu ruangan hanya bisa dipakai satu kali perkuliahan dalam satu waktu.
\end{enumerate} 
Kemudian dilakukan variasi nilai probabilitas mutasi dimana masing-masing nilai tersebut dilakukan 5 kali perulangan. 
Berdasarkan pengujian, diperoleh rata-rata waktu seperti pada Tabel \ref{tab:3 constraint}

\begin{longtable}[c]{|c|c|}
  \caption{Rata-rata waktu proses penjadwalan dengan 3 constraint}
  \label{tab:3 constraint}\\
  \hline
  \rowcolor[HTML]{C0C0C0} 
  Probabilitas Mutasi & Rata-rata waktu \\ \hline
  0.02                & 10m 23s         \\ \hline
  0.04                & 5m 27s          \\ \hline
  0.06                & 4m 41s          \\ \hline
  0.08                & 7m 45s          \\ \hline
  0.1                 & 9m 3s           \\ \hline
\end{longtable}
Berdasarkan data pada Tabel \ref{tab:3 constraint}, semakin besar nilai probabilitas mutasi, maka waktu penyelesaian program akan semakin lama, selain itu ketika nilai probabilitas mutasi mendekati 0, waktu penyelesaian program juga semakin lama.
Contoh hasil keluaran program pada pengujian ini adalah sebagai berikut:
  \begin{longtable}[c]{|c|c|>{\centering\arraybackslash}m{2.5cm}|>{\centering\arraybackslash}m{4cm}|>{\centering\arraybackslash}m{3.2cm}|}
    \caption{Hasil Penjadwalan Otomatis dengan 3 Constraints}
    \label{tab:3}\\
    \hline
    \rowcolor[HTML]{C0C0C0} 
    Hari   & Waktu       & Ruang                  & Matkul                                                              & Dosen                                         \\ \hline
    Senin  & 07.30-10.00 & AJ402, Kapasitas: 25   & Rangkaian Listrik A, \linebreak Peserta: 40                         & Ir. Hany Boedinugroho, M.T.                   \\ \hline
    Senin  & 07.30-10.00 & AJ401, Kapasitas: 25   & Dasar Pemrograman P, \linebreak Peserta: 14                         & Dr. Surya Sumpeno, S.T.,M.Sc.                 \\ \hline
    Senin  & 07.30-10.00 & A108, Kapasitas: 80    & Matematika Diskrit A, \linebreak Peserta: 32                        & Prof. Dr. Ir. Mauridhi Hery Purnomo, M.Eng.   \\ \hline
    Senin  & 10.00-12.30 & B211, Kapasitas: 25    & Pemrograman Sistem dan Jaringan A, \linebreak Peserta: 11           & Reza Fuad Rachmadi, S.T.,M.T.,Ph.D            \\ \hline
    Senin  & 13.30-16.00 & Pasca 2, Kapasitas: 40 & Jaringan Komputer B, \linebreak Peserta: 40                         & Dr. Supeno Mardi Susiki Nugroho, S.T., M.T.   \\ \hline
    Senin  & 13.30-16.00 & AJ401, Kapasitas: 25   & Desain Pemrograman Game A, \linebreak Peserta: 9                    & Ir. Hany Boedinugroho, M.T.                   \\ \hline
    Senin  & 13.30-16.00 & A235, Kapasitas: 40    & Jaringan Komputer A, \linebreak Peserta: 41                         & Dion Hayu Fandiantoro, S.T., M.Eng.           \\ \hline
    Selasa & 07.30-10.00 & AJ401, Kapasitas: 25   & Rangkaian Listrik B, \linebreak Peserta: 35                         & Eko Pramunanto, S.T., M.T.                    \\ \hline
    Selasa & 07.30-10.00 & B211, Kapasitas: 25    & Arsitektur dan Organisasi Sistem Komputer A, \linebreak Peserta: 30 & Dion Hayu Fandiantoro, S.T., M.Eng.           \\ \hline
    Selasa & 07.30-10.00 & A108, Kapasitas: 80    & Pengolahan Sinyal Digital A, \linebreak Peserta: 26                 & Dr. Diah Puspito Wulandari, S.T.,M.Sc.        \\ \hline
    Selasa & 10.00-12.30 & Pasca 2, Kapasitas: 40 & Sistem Tertanam B, \linebreak Peserta: 24                           & Muhtadin, S.T., M.T.                          \\ \hline
    Selasa & 10.00-12.30 & A234, Kapasitas: 36    & Aljabar Linear A, \linebreak Peserta: 5                             & Dr. Diah Puspito Wulandari, S.T.,M.Sc.        \\ \hline
    Selasa & 10.00-12.30 & A108, Kapasitas: 80    & Sistem Operasi A, \linebreak Peserta: 25                            & Susi Juniastuti, S.T.,M.Eng.                  \\ \hline
    Selasa & 10.00-12.30 & AJ402, Kapasitas: 25   & Desain Dan Rekayasa Sistem B, \linebreak Peserta: 40                & Mochamad Hariadi, S.T.,M.Sc.,Ph.D             \\ \hline
    Selasa & 13.30-16.00 & A234, Kapasitas: 36    & Pengantar Robotika A, \linebreak Peserta: 33                        & Dion Hayu Fandiantoro, S.T., M.Eng.           \\ \hline
    Rabu   & 07.30-10.00 & A108, Kapasitas: 80    & Desain Dan Rekayasa Sistem A, \linebreak Peserta: 49                & Atar Fuady Babgei, S.T., M.Sc.                \\ \hline
    Rabu   & 07.30-10.00 & Pasca 2, Kapasitas: 40 & Persamaan Differensial dan Deret A, \linebreak Peserta: 35          & Muhtadin, S.T., M.T.                          \\ \hline
    Rabu   & 07.30-10.00 & AJ401, Kapasitas: 25   & Sekuriti Sistem Komputer A, \linebreak Peserta: 33                  & Arief Kurniawan, S.T., M.T.                   \\ \hline
    Rabu   & 10.00-12.30 & Pasca 2, Kapasitas: 40 & Sistem Manajemen Basis Data A, \linebreak Peserta: 25               & Reza Fuad Rachmadi, S.T.,M.T.,Ph.D            \\ \hline
    Rabu   & 10.00-12.30 & A235, Kapasitas: 40    & Sistem Mikroprosesor dan Mikrokontroller A, \linebreak Peserta: 26  & Arief Kurniawan, S.T., M.T.                   \\ \hline
    Rabu   & 13.30-16.00 & A235, Kapasitas: 40    & Sistem Operasi B, \linebreak Peserta: 26                            & Mochamad Hariadi, S.T.,M.Sc.,Ph.D             \\ \hline
    Rabu   & 13.30-16.00 & A108, Kapasitas: 80    & Metode Numerik C, \linebreak Peserta: 59                            & Prof. Dr. Ir. Yoyon Kusnendar Suprapto, M.Sc. \\ \hline
    Rabu   & 13.30-16.00 & B211, Kapasitas: 25    & Deep Learning untuk Multimedia A, \linebreak Peserta: 30            & Eko Pramunanto, S.T., M.T.                    \\ \hline
    Kamis  & 07.30-10.00 & A234, Kapasitas: 36    & Matematika Diskrit B, \linebreak Peserta: 59                        & Dr. Eko Mulyanto Yuniarno, S.T.,M.T.          \\ \hline
    Kamis  & 07.30-10.00 & B211, Kapasitas: 25    & Arsitektur dan Organisasi Sistem Komputer B, \linebreak Peserta: 27 & Dr. Diah Puspito Wulandari, S.T.,M.Sc.        \\ \hline
    Kamis  & 07.30-10.00 & A235, Kapasitas: 40    & Sistem Mikroprosesor dan Mikrokontroller B, \linebreak Peserta: 20  & Atar Fuady Babgei, S.T., M.Sc.                \\ \hline
    Kamis  & 07.30-10.00 & A108, Kapasitas: 80    & Pemrograman Lanjut B, \linebreak Peserta: 39                        & Dr. Surya Sumpeno, S.T.,M.Sc.                 \\ \hline
    Kamis  & 07.30-10.00 & Pasca 2, Kapasitas: 40 & Metode Numerik A, \linebreak Peserta: 40                            & Reza Fuad Rachmadi, S.T.,M.T.,Ph.D            \\ \hline
    Kamis  & 07.30-10.00 & AJ402, Kapasitas: 25   & Pemrograman Lanjut A, \linebreak Peserta: 39                        & Susi Juniastuti, S.T.,M.Eng.                  \\ \hline
    Kamis  & 10.00-12.30 & B211, Kapasitas: 25    & Desain Pemrograman Game P, \linebreak Peserta: 8                    & Dr. I Ketut Eddy Purnama, S.T.,M.T.           \\ \hline
    Kamis  & 10.00-12.30 & A234, Kapasitas: 36    & Visi Komputer A, \linebreak Peserta: 16                             & Dr. Diah Puspito Wulandari, S.T.,M.Sc.        \\ \hline
    Kamis  & 10.00-12.30 & A235, Kapasitas: 40    & Sistem Manajemen Basis Data B, \linebreak Peserta: 25               & Dr. Surya Sumpeno, S.T.,M.Sc.                 \\ \hline
    Jumat  & 07.30-10.00 & A235, Kapasitas: 40    & Sistem Tertanam A, \linebreak Peserta: 27                           & Dr. Surya Sumpeno, S.T.,M.Sc.                 \\ \hline
    Jumat  & 07.30-10.00 & AJ401, Kapasitas: 25   & Pengolahan Sinyal Digital B, \linebreak Peserta: 26                 & Ahmad Zaini, S.T., M.Sc.                      \\ \hline
    Jumat  & 07.30-10.00 & A234, Kapasitas: 36    & Persepsi Robot A, \linebreak Peserta: 13                            & Mochamad Hariadi, S.T.,M.Sc.,Ph.D             \\ \hline
    Jumat  & 13.30-16.00 & AJ401, Kapasitas: 25   & Rangkaian Digital A, \linebreak Peserta: 24                         & Atar Fuady Babgei, S.T., M.Sc.                \\ \hline
    Jumat  & 13.30-16.00 & AJ402, Kapasitas: 25   & Probabilitas dan Statistik A, \linebreak Peserta: 28                & Reza Fuad Rachmadi, S.T.,M.T.,Ph.D            \\ \hline
    Jumat  & 13.30-16.00 & B211, Kapasitas: 25    & Metode Numerik B, \linebreak Peserta: 40                            & Ir. Hany Boedinugroho, M.T.                   \\ \hline
  \end{longtable}
Jadwal yang dihasilkan pada Tabel \ref{tab:3} masih memiliki ketidaksesuaian karena terdapat jumlah peserta perkuliahan yang melebihi kapasitas ruangan sehingga jadwal yang dihasilkan pada pengujian ini tidak bisa dipakai untuk perkuliahan.
\section{Percobaan dengan 4 constraint}
\label{sec:pengujian 4}
Pada pengujian ini yang dilakukan adalah melakukan perhitungan nilai fitness menggunakan 3 constraint yaitu:
\begin{enumerate}[nolistsep]
  \item Satu mata kuliah hanya bisa diselenggarakan satu kali dalam satu pekan perkuliahan.
  \item Satu dosen hanya bisa mengajar satu kali perkuliahan dalam satu waktu.
  \item Satu ruangan hanya bisa dipakai satu kali perkuliahan dalam satu waktu.
  \item Jumlah peserta perkuliahan tidak boleh melebihi kapasitas ruangan
\end{enumerate} 
Kemudian dilakukan variasi nilai probabilitas mutasi dimana masing-masing nilai tersebut dilakukan 5 kali perulangan. 
Berdasarkan pengujian, diperoleh rata-rata waktu seperti pada Tabel \ref{tab:3 constraint}

\begin{longtable}[c]{|c|c|}
  \caption{Rata-rata waktu proses penjadwalan dengan 4 constraint}
  \label{tab:4 constaint}\\
  \hline
  \rowcolor[HTML]{C0C0C0} 
  Probabilitas Mutasi & Rata-rata waktu \\ \hline
  \endfirsthead
  %
  \endhead
  %
  0.02                & 1h 12m 29s      \\ \hline
  0.04                & 38m 41s         \\ \hline
  0.06                & 38m 27s         \\ \hline
  0.08                & 38m 56s         \\ \hline
  0.1                 & 57m 23s         \\ \hline
  \end{longtable}

\begin{longtable}[c]{|c|c|>{\centering\arraybackslash}m{2.5cm}|>{\centering\arraybackslash}m{4cm}|>{\centering\arraybackslash}m{3.2cm}|}
  \caption{Hasil Penjadwalan Otomatis dengan 4 Constraints}
  \label{tab:4}\\
  \hline
  \rowcolor[HTML]{C0C0C0} 
  Hari   & Waktu       & Ruang                  & Matkul                                                              & Dosen                                         \\ \hline
  Senin  & 07.30-10.00 & Pasca 2, Kapasitas: 40 & Visi Komputer A, \linebreak Peserta: 16                             & Arief Kurniawan, S.T., M.T.                   \\ \hline
  Senin  & 07.30-10.00 & A108, Kapasitas: 80    & Sistem Manajemen Basis Data A, \linebreak Peserta: 25               & Dr. Surya Sumpeno, S.T.,M.Sc.                 \\ \hline
  Senin  & 07.30-10.00 & A235, Kapasitas: 40    & Arsitektur dan Organisasi Sistem Komputer B, \linebreak Peserta: 27 & Atar Fuady Babgei, S.T., M.Sc.                \\ \hline
  Senin  & 10.00-12.30 & A235, Kapasitas: 40    & Jaringan Komputer B, \linebreak Peserta: 40                         & Prof. Dr. Ir. Yoyon Kusnendar Suprapto, M.Sc. \\ \hline
  Senin  & 10.00-12.30 & Pasca 2, Kapasitas: 40 & Matematika Diskrit A, \linebreak Peserta: 32                        & Dr. Supeno Mardi Susiki Nugroho, S.T., M.T.   \\ \hline
  Senin  & 13.30-16.00 & A235, Kapasitas: 40    & Rangkaian Listrik A, \linebreak Peserta: 40                         & Dr. Supeno Mardi Susiki Nugroho, S.T., M.T.   \\ \hline
  Senin  & 13.30-16.00 & Pasca 2, Kapasitas: 40 & Sistem Tertanam A, \linebreak Peserta: 27                           & Ir. Hany Boedinugroho, M.T.                   \\ \hline
  Selasa & 07.30-10.00 & B211, Kapasitas: 25    & Desain Pemrograman Game A, \linebreak Peserta: 9                    & Muhtadin, S.T., M.T.                          \\ \hline
  Selasa & 07.30-10.00 & A108, Kapasitas: 80    & Desain Dan Rekayasa Sistem A, \linebreak Peserta: 49                & Susi Juniastuti, S.T.,M.Eng.                  \\ \hline
  Selasa & 07.30-10.00 & Pasca 2, Kapasitas: 40 & Persamaan Differensial dan Deret A, \linebreak Peserta: 35          & Prof. Dr. Ir. Yoyon Kusnendar Suprapto, M.Sc. \\ \hline
  Selasa & 07.30-10.00 & A235, Kapasitas: 40    & Pengantar Robotika A, \linebreak Peserta: 33                        & Reza Fuad Rachmadi, S.T.,M.T.,Ph.D            \\ \hline
  Selasa & 10.00-12.30 & A234, Kapasitas: 36    & Sistem Mikroprosesor dan Mikrokontroller B, \linebreak Peserta: 20  & Dr. Supeno Mardi Susiki Nugroho, S.T., M.T.   \\ \hline
  Selasa & 10.00-12.30 & A108, Kapasitas: 80    & Jaringan Komputer A, \linebreak Peserta: 41                         & Dr. Eko Mulyanto Yuniarno, S.T.,M.T.          \\ \hline
  Selasa & 13.30-16.00 & A235, Kapasitas: 40    & Probabilitas dan Statistik A, \linebreak Peserta: 28                & Dr. Supeno Mardi Susiki Nugroho, S.T., M.T.   \\ \hline
  Selasa & 13.30-16.00 & A108, Kapasitas: 80    & Matematika Diskrit B, \linebreak Peserta: 59                        & Mochamad Hariadi, S.T.,M.Sc.,Ph.D             \\ \hline
  Selasa & 13.30-16.00 & Pasca 2, Kapasitas: 40 & Pengolahan Sinyal Digital B, \linebreak Peserta: 26                 & Susi Juniastuti, S.T.,M.Eng.                  \\ \hline
  Rabu   & 07.30-10.00 & A235, Kapasitas: 40    & Arsitektur dan Organisasi Sistem Komputer A, \linebreak Peserta: 30 & Dr. Diah Puspito Wulandari, S.T.,M.Sc.        \\ \hline
  Rabu   & 10.00-12.30 & A234, Kapasitas: 36    & Sistem Operasi B, \linebreak Peserta: 26                            & Ahmad Zaini, S.T., M.Sc.                      \\ \hline
  Rabu   & 10.00-12.30 & A108, Kapasitas: 80    & Rangkaian Listrik B, \linebreak Peserta: 35                         & Eko Pramunanto, S.T., M.T.                    \\ \hline
  Rabu   & 10.00-12.30 & A235, Kapasitas: 40    & Sistem Operasi A, \linebreak Peserta: 25                            & Arief Kurniawan, S.T., M.T.                   \\ \hline
  Rabu   & 13.30-16.00 & Pasca 2, Kapasitas: 40 & Pemrograman Lanjut A, \linebreak Peserta: 39                        & Susi Juniastuti, S.T.,M.Eng.                  \\ \hline
  Rabu   & 13.30-16.00 & A108, Kapasitas: 80    & Aljabar Linear A, \linebreak Peserta: 5                             & Reza Fuad Rachmadi, S.T.,M.T.,Ph.D            \\ \hline
  Rabu   & 13.30-16.00 & A234, Kapasitas: 36    & Deep Learning untuk Multimedia A, \linebreak Peserta: 30            & Dr. Eko Mulyanto Yuniarno, S.T.,M.T.          \\ \hline
  Kamis  & 07.30-10.00 & A108, Kapasitas: 80    & Sekuriti Sistem Komputer A, \linebreak Peserta: 33                  & Prof. Dr. Ir. Yoyon Kusnendar Suprapto, M.Sc. \\ \hline
  Kamis  & 07.30-10.00 & AJ402, Kapasitas: 25   & Sistem Manajemen Basis Data B, \linebreak Peserta: 25               & Arief Kurniawan, S.T., M.T.                   \\ \hline
  Kamis  & 07.30-10.00 & B211, Kapasitas: 25    & Sistem Operasi A, \linebreak Peserta: 25                            & Dr. Supeno Mardi Susiki Nugroho, S.T., M.T.   \\ \hline
  Kamis  & 07.30-10.00 & A234, Kapasitas: 36    & Pemrograman Sistem dan Jaringan A, \linebreak Peserta: 11           & Ir. Hany Boedinugroho, M.T.                   \\ \hline
  Kamis  & 07.30-10.00 & A235, Kapasitas: 40    & Pengolahan Sinyal Digital A, \linebreak Peserta: 26                 & Reza Fuad Rachmadi, S.T.,M.T.,Ph.D            \\ \hline
  Kamis  & 10.00-12.30 & A234, Kapasitas: 36    & Sistem Tertanam B, \linebreak Peserta: 24                           & Dr. Diah Puspito Wulandari, S.T.,M.Sc.        \\ \hline
  Kamis  & 10.00-12.30 & A235, Kapasitas: 40    & Pemrograman Lanjut B, \linebreak Peserta: 39                        & Dr. Supeno Mardi Susiki Nugroho, S.T., M.T.   \\ \hline
  Kamis  & 10.00-12.30 & AJ401, Kapasitas: 25   & Desain Dan Rekayasa Sistem B, \linebreak Peserta: 40                & Ir. Hany Boedinugroho, M.T.                   \\ \hline
  Kamis  & 13.30-16.00 & Pasca 2, Kapasitas: 40 & Sistem Mikroprosesor dan Mikrokontroller A, \linebreak Peserta: 26  & Dion Hayu Fandiantoro, S.T., M.Eng.           \\ \hline
  Kamis  & 13.30-16.00 & AJ402, Kapasitas: 25   & Dasar Pemrograman P, \linebreak Peserta: 14                         & Dr. Surya Sumpeno, S.T.,M.Sc.                 \\ \hline
  Kamis  & 13.30-16.00 & B211, Kapasitas: 25    & Desain Pemrograman Game P, \linebreak Peserta: 8                    & Susi Juniastuti, S.T.,M.Eng.                  \\ \hline
  Jumat  & 07.30-10.00 & Pasca 2, Kapasitas: 40 & Metode Numerik A, \linebreak Peserta: 40                            & Muhtadin, S.T., M.T.                          \\ \hline
  Jumat  & 07.30-10.00 & A108, Kapasitas: 80    & Persepsi Robot A, \linebreak Peserta: 13                            & Dion Hayu Fandiantoro, S.T., M.Eng.           \\ \hline
  Jumat  & 13.30-16.00 & AJ401, Kapasitas: 25   & Rangkaian Digital A, \linebreak Peserta: 24                         & Ahmad Zaini, S.T., M.Sc.                      \\ \hline
  Jumat  & 13.30-16.00 & A108, Kapasitas: 80    & Metode Numerik B, \linebreak Peserta: 40                            & Dr. Eko Mulyanto Yuniarno, S.T.,M.T.          \\ \hline
\end{longtable}

% \section{Hasil Penelitian}
% \label{sec:hasil penelitian}
% Sistem yang sudah dibuat mampu menghasilkan individu yang memenuhi batasan-batasan yang telah ditentukan sebelumnya dan individu terbaik dapat menghasilkan nilai fitness 1.00. Nilai fitness 1.00 berarti tidak ada batasan yang dilanggar. Contoh hasil keluaran program penjadwalan otomatis ini adalah sebagai berikut:
 
% % \begin{lstlisting}[language=Python,caption={Output penjadwalan otomatis}]
% %   Generation:  105831
% %   3
% %   Mutate
% %   Fitness a: 1.0
% %   Fitness b: 1.0
% %   Chromosome terbaik: 
% %   [[10, 2, 3, 1], [9, 26, 4, 1], [17, 9, 5, 1], [11, 5, 2, 3], [7, 18, 3, 3], [6, 12, 5, 3], [10, 25, 3, 4], [15, 32, 4, 4], [12, 28, 5, 4], [2, 35, 2, 5], [3, 16, 4, 5], [9, 10, 6, 5], [9, 6, 7, 6], [14, 1, 1, 6], [17, 14, 3, 6], [2, 4, 2, 6], [10, 27, 4, 6], [13, 37, 1, 7], [4, 33, 7, 7], [5, 22, 5, 7], [1, 11, 4, 7], [11, 21, 2, 8], [7, 13, 4, 8], [3, 30, 7, 8], [2, 24, 5, 9], [9, 29, 4, 10], [17, 20, 4, 11], [5, 7, 6, 11], [16, 17, 3, 12], [11, 38, 6, 12], [4, 3, 5, 12], [2, 8, 4, 12], [6, 15, 4, 13], [11, 34, 2, 14], [3, 36, 4, 14], [14, 31, 7, 14], [13, 19, 6, 14], [2, 23, 3, 14]]
% % \end{lstlisting}
% \begin{longtable}[c]{|c|c|c|>{\centering\arraybackslash}m{3cm}|>{\centering\arraybackslash}m{3cm}|}
%   \caption{Hasil Penjadwalan Otomatis}
%   \label{tab:jadwal}\\
%   \hline
%   Hari   & Waktu       & Ruang                  & Matkul                                                              & Dosen                                         \\ \hline
%   \rowcolor[HTML]{DAE8FC} 
%   Senin  & 07.30-10.00 & A235, Kapasitas: 40    & Aljabar Linear A, \linebreak Peserta: 5                             & Muhtadin, S.T., M.T.                          \\ \hline
%   \rowcolor[HTML]{DAE8FC} 
%   Senin  & 07.30-10.00 & A108, Kapasitas: 80    & Rangkaian Digital A, \linebreak Peserta: 24                         & Arief Kurniawan, S.T., M.T.                   \\ \hline
%   \rowcolor[HTML]{DAE8FC} 
%   Senin  & 07.30-10.00 & Pasca 2, Kapasitas: 40 & Desain Dan Rekayasa Sistem B, \linebreak Peserta: 40                & Dion Hayu Fandiantoro, S.T., M.Eng.           \\ \hline
%   \rowcolor[HTML]{DAE8FC} 
%   Senin  & 13.30-16.00 & A234, Kapasitas: 36    & Deep Learning untuk Multimedia A, \linebreak Peserta: 30            & Atar Fuady Babgei, S.T., M.Sc.                \\ \hline
%   \rowcolor[HTML]{DAE8FC} 
%   Senin  & 13.30-16.00 & A235, Kapasitas: 40    & Pemrograman Lanjut B, \linebreak Peserta: 39                        & Dr. Surya Sumpeno, S.T.,M.Sc.                 \\ \hline
%   \rowcolor[HTML]{DAE8FC} 
%   Senin  & 13.30-16.00 & Pasca 2, Kapasitas: 40 & Matematika Diskrit A, \linebreak Peserta: 32                        & Prof. Dr. Ir. Mauridhi Hery Purnomo, M.Eng.   \\ \hline
%   Selasa & 07.30-10.00 & A235, Kapasitas: 40    & Probabilitas dan Statistik A, \linebreak Peserta: 28                & Muhtadin, S.T., M.T.                          \\ \hline
%   Selasa & 07.30-10.00 & A108, Kapasitas: 80    & Sistem Mikroprosesor dan Mikrokontroller A, \linebreak Peserta: 26  & Ahmad Zaini, S.T., M.Sc.                      \\ \hline
%   Selasa & 07.30-10.00 & Pasca 2, Kapasitas: 40 & Rangkaian Listrik B, \linebreak Peserta: 35                         & Mochamad Hariadi, S.T.,M.Sc.,Ph.D             \\ \hline
%   Selasa & 10.00-12.30 & A234, Kapasitas: 36    & Sistem Operasi B, \linebreak Peserta: 26                            & Dr. Diah Puspito Wulandari, S.T.,M.Sc.        \\ \hline
%   Selasa & 10.00-12.30 & A108, Kapasitas: 80    & Metode Numerik C, \linebreak Peserta: 59                            & Dr. I Ketut Eddy Purnama, S.T.,M.T.           \\ \hline
%   Selasa & 10.00-12.30 & AJ401, Kapasitas: 25   & Jaringan Komputer A, \linebreak Peserta: 41                         & Arief Kurniawan, S.T., M.T.                   \\ \hline
%   Selasa & 13.30-16.00 & AJ402, Kapasitas: 25   & Desain Pemrograman Game A, \linebreak Peserta: 9                    & Arief Kurniawan, S.T., M.T.                   \\ \hline
%   Selasa & 13.30-16.00 & B211, Kapasitas: 25    & Dasar Pemrograman P, \linebreak Peserta: 14                         & Susi Juniastuti, S.T.,M.Eng.                  \\ \hline
%   Selasa & 13.30-16.00 & A235, Kapasitas: 40    & Metode Numerik A, \linebreak Peserta: 40                            & Dion Hayu Fandiantoro, S.T., M.Eng.           \\ \hline
%   Selasa & 13.30-16.00 & A234, Kapasitas: 36    & Arsitektur dan Organisasi Sistem Komputer B, \linebreak Peserta: 27 & Dr. Diah Puspito Wulandari, S.T.,M.Sc.        \\ \hline
%   Selasa & 13.30-16.00 & A108, Kapasitas: 80    & Rangkaian Listrik A, \linebreak Peserta: 40                         & Muhtadin, S.T., M.T.                          \\ \hline
%   \rowcolor[HTML]{DAE8FC} 
%   Rabu   & 07.30-10.00 & B211, Kapasitas: 25    & Sistem Tertanam B, \linebreak Peserta: 24                           & Dr. Supeno Mardi Susiki Nugroho, S.T., M.T.   \\ \hline
%   \rowcolor[HTML]{DAE8FC} 
%   Rabu   & 07.30-10.00 & AJ402, Kapasitas: 25   & Sistem Mikroprosesor dan Mikrokontroller B, \linebreak Peserta: 20  & Ir. Hany Boedinugroho, M.T.                   \\ \hline
%   \rowcolor[HTML]{DAE8FC} 
%   Rabu   & 07.30-10.00 & Pasca 2, Kapasitas: 40 & Pengolahan Sinyal Digital B, \linebreak Peserta: 26                 & Prof. Dr. Ir. Yoyon Kusnendar Suprapto, M.Sc. \\ \hline
%   \rowcolor[HTML]{DAE8FC} 
%   Rabu   & 07.30-10.00 & A108, Kapasitas: 80    & Jaringan Komputer B, \linebreak Peserta: 40                         & Dr. Eko Mulyanto Yuniarno, S.T.,M.T.          \\ \hline
%   \rowcolor[HTML]{DAE8FC} 
%   Rabu   & 10.00-12.30 & A234, Kapasitas: 36    & Pengolahan Sinyal Digital A, \linebreak Peserta: 26                 & Atar Fuady Babgei, S.T., M.Sc.                \\ \hline
%   \rowcolor[HTML]{DAE8FC} 
%   Rabu   & 10.00-12.30 & A108, Kapasitas: 80    & Matematika Diskrit B, \linebreak Peserta: 59                        & Dr. Surya Sumpeno, S.T.,M.Sc.                 \\ \hline
%   \rowcolor[HTML]{DAE8FC} 
%   Rabu   & 10.00-12.30 & AJ402, Kapasitas: 25   & Sistem Manajemen Basis Data A, \linebreak Peserta: 25               & Dr. I Ketut Eddy Purnama, S.T.,M.T.           \\ \hline
%   \rowcolor[HTML]{DAE8FC} 
%   Rabu   & 13.30-16.00 & Pasca 2, Kapasitas: 40 & Persepsi Robot A, \linebreak Peserta: 13                            & Dr. Diah Puspito Wulandari, S.T.,M.Sc.        \\ \hline
%   Kamis  & 07.30-10.00 & A108, Kapasitas: 80    & Sekuriti Sistem Komputer A, \linebreak Peserta: 33                  & Arief Kurniawan, S.T., M.T.                   \\ \hline
%   Kamis  & 10.00-12.30 & A108, Kapasitas: 80    & Pengantar Robotika A, \linebreak Peserta: 33                        & Dion Hayu Fandiantoro, S.T., M.Eng.           \\ \hline
%   Kamis  & 10.00-12.30 & AJ401, Kapasitas: 25   & Desain Pemrograman Game P, \linebreak Peserta: 8                    & Prof. Dr. Ir. Yoyon Kusnendar Suprapto, M.Sc. \\ \hline
%   Kamis  & 13.30-16.00 & A235, Kapasitas: 40    & Pemrograman Lanjut A, \linebreak Peserta: 39                        & Reza Fuad Rachmadi, S.T.,M.T.,Ph.D            \\ \hline
%   Kamis  & 13.30-16.00 & AJ401, Kapasitas: 25   & Visi Komputer A, \linebreak Peserta: 16                             & Atar Fuady Babgei, S.T., M.Sc.                \\ \hline
%   Kamis  & 13.30-16.00 & Pasca 2, Kapasitas: 40 & Arsitektur dan Organisasi Sistem Komputer A, \linebreak Peserta: 30 & Ir. Hany Boedinugroho, M.T.                   \\ \hline
%   Kamis  & 13.30-16.00 & A108, Kapasitas: 80    & Desain Dan Rekayasa Sistem A, \linebreak Peserta: 49                & Dr. Diah Puspito Wulandari, S.T.,M.Sc.        \\ \hline
%   \rowcolor[HTML]{DAE8FC} 
%   Jumat  & 07.30-10.00 & A108, Kapasitas: 80    & Metode Numerik B, \linebreak Peserta: 40                            & Prof. Dr. Ir. Mauridhi Hery Purnomo, M.Eng.   \\ \hline
%   \rowcolor[HTML]{DAE8FC} 
%   Jumat  & 13.30-16.00 & A234, Kapasitas: 36    & Sistem Operasi A, \linebreak Peserta: 25                            & Atar Fuady Babgei, S.T., M.Sc.                \\ \hline
%   \rowcolor[HTML]{DAE8FC} 
%   Jumat  & 13.30-16.00 & A108, Kapasitas: 80    & Sistem Tertanam A, \linebreak Peserta: 27                           & Dr. I Ketut Eddy Purnama, S.T.,M.T.           \\ \hline
%   \rowcolor[HTML]{DAE8FC} 
%   Jumat  & 13.30-16.00 & AJ402, Kapasitas: 25   & Sistem Manajemen Basis Data B, \linebreak Peserta: 25               & Susi Juniastuti, S.T.,M.Eng.                  \\ \hline
%   \rowcolor[HTML]{DAE8FC} 
%   Jumat  & 13.30-16.00 & AJ401, Kapasitas: 25   & Pemrograman Sistem dan Jaringan A, \linebreak Peserta: 11           & Dr. Supeno Mardi Susiki Nugroho, S.T., M.T.   \\ \hline
%   \rowcolor[HTML]{DAE8FC} 
%   Jumat  & 13.30-16.00 & A235, Kapasitas: 40    & Persamaan Differensial dan Deret A, \linebreak Peserta: 35          & Dr. Diah Puspito Wulandari, S.T.,M.Sc.        \\ \hline
% \end{longtable}
% Hasil keluaran program diatas merupakan hasil keluaran dari percobaan penjadwalan otomatis yang memperhatikan kapasitas ruangan dan jumlah peserta tiap mata kuliah dan menggunakan probabilitasan percobaan yang memperhatikan probabilitasis mutasi sebesar 0.05. Percobaan ini memakan waktu 14 menit 52 detik.

% Terdapat dua percobaan yang dilakukan dalam sistem penjadwalan otomatis ini. Percobaan pertama yaitu tanpa memperhatikan kapasitas kelas dan jumlah perserta tiap mata kuliah, dan percobaan kedua dengan memperhatikan kapasitas ruang kelas dan jumlah perserta tiap mata kuliah.
% Masing-masing percobaan dilakukan dengan memvariasikan probabilitas mutasi yaitu dengan nilai 0.04, 0.05, dan 0.06.
% Masing-masing percobaan dilakukan dengan 5 kali \emph{loop} pada konfigurasi yang sama. 
% Hasil percobaan yang telah dilakukan dapat dilihat pada tabel dibawah ini:
% \begin{longtable}{|c|c|}
%   \caption{Percobaan tanpa memperhatikan kapasitas \linebreak ruang dan jumlah peserta mata kuliah}
%   \label{tab:tanpaKapasitas}\\
%   \hline
%   \rowcolor[HTML]{C0C0C0} 
% {\color[HTML]{000000} \textbf{Probabilitas Mutasi}} & {\color[HTML]{000000} \textbf{Rata-Rata Waktu}} \\ \hline
% 0.04                                                & 6m 46s                                           \\ \hline
% 0.05                                                & 5m 38s                                           \\ \hline
% 0.06                                                & 4m 54s                                           \\ \hline
% \end{longtable}

% \begin{longtable}{|c|c|}
%   \caption{Percobaan dengan memperhatikan kapasitas \linebreak ruang dan jumlah peserta mata kuliah}
%   \label{tab:denganKapasitas}\\
%   \hline
%   \rowcolor[HTML]{C0C0C0} 
% {\color[HTML]{000000} \textbf{Probabilitas Mutasi}} & {\color[HTML]{000000} \textbf{Rata-Rata Waktu}} \\ \hline
% 0.04                                                & 39m 43s                                         \\ \hline
% 0.05                                                & 14m 53s                                         \\ \hline
% 0.06                                                & 23m 18s                                         \\ \hline
% \end{longtable}


% % Pengujian dilakukan dengan memvariasikan nilai probabilitas mutasi yang kemudian diperoleh rataan waktu untuk menyelesaikan proses penjadwalan. 
% % Tabel rata-rata waktu proses penjadwalan adalah sebagai berikut.
% % \begin{longtable}{|c|c|}
% %   \caption{Hasil Percobaan Variasi Probabilitas Mutasi}
% %   \label{tab:variasiMutasi}\\
% %   \hline
% %   \rowcolor[HTML]{C0C0C0} 
% %   {\color[HTML]{000000} \textbf{Probabilitas Mutasi}} & {\color[HTML]{000000} \textbf{Rata-Rata Waktu}} \\ \hline
% %   0.04                                                & 39m 43s                                         \\ \hline
% %   0.05                                                & 14m 53s                                         \\ \hline
% %   0.06                                                & 23m 18s                                         \\ \hline
% % \end{longtable}
% \section{Pembahasan}
% \label{sec:Pembahasan}

% Dari pengujian yang telah dilakukan dapat diketahui bahwa semakin banyak batasan-batasan yang diberikan dalam proses algoritma genetika, maka kompleksitas dari proses perhitungan evaluasi fitness juga akan semakin rumit. 
% Kerumitan proses perhitungan ini tentu akan menambah durasi dari proses algoritma genetika itu sendiri. Hal ini dapat dilihat dari perbedaan waktu antara Tabel \ref{tab:tanpaKapasitas} dengan Tabel \ref{tab:denganKapasitas}.

% Selain kompleksitas perhitungan evaluasi fitness dari satu individu, penentuan probabilitas mutasi juga sangat berpengaruh terhadap durasi proses algoritma genetika. Probabilitas \linebreak mutasi dalam sebuah proses algoritma genetika tidak boleh terlalu besar, karena mengakibatkan hilangnya kemiripan antara individu baru dengan individu induknya. 
% Sedangkan jika probabilitas mutasi bernilai terlalu kecil, maka durasi untuk memperoleh solusi berupa individu terbaik, akan memakan waktu yang sangat lama dan membuat proses algorima genetika dari penjadwalan otomatis ini menjadi tidak optimal. 

% Data yang diperoleh pada Tabel \ref{tab:tanpaKapasitas} menunjukkan bahwa probabilitas mutasi 0.06 memiliki rata-rata durasi yang paling singkat yaitu dengan waktu 4 menit 54 detik. Sedangkan data pada Tabel \ref{tab:denganKapasitas} menunjukkan bahwa probabilitas mutasi 0.05 memiliki rata-rata durasi yang paling singkat yakni dengan waktu 14 menit 53 detik. 
% % Contoh pembuatan tabel
% % \begin{longtable}{|c|c|c|}
% %   \caption{Hasil Pengukuran Energi dan Kecepatan}
% %   \label{tb:EnergiKecepatan}                                   \\
% %   \hline
% %   \rowcolor[HTML]{C0C0C0}
% %   \textbf{Energi} & \textbf{Jarak Tempuh} & \textbf{Kecepatan} \\
% %   \hline
% %   10 J            & 1000 M                & 200 M/s            \\
% %   20 J            & 2000 M                & 400 M/s            \\
% %   30 J            & 4000 M                & 800 M/s            \\
% %   40 J            & 8000 M                & 1600 M/s           \\
% %   \hline
% % \end{longtable}

% % \lipsum[2-4]

\chapter{HASIL DAN PEMBAHASAN}
\label{chap:hasilpembahasan}

% Ubah bagian-bagian berikut dengan isi dari pengujian dan analisis

% Pada penelitian ini dipaparkan \lipsum[1][1-5]

\section{Hasil Penelitian}
\label{sec:hasil penelitian}
Sistem yang sudah dibuat mampu menghasilkan individu yang memenuhi batasan-batasan yang telah ditentukan sebelumnya dan individu terbaik dapat menghasilkan nilai fitness 1.00. Nilai fitness 1.00 berarti tidak ada batasan yang dilanggar. Contoh hasil keluaran program penjadwalan otomatis ini adalah sebagai berikut:
 
% \begin{lstlisting}[language=Python,caption={Output penjadwalan otomatis}]
%   Generation:  105831
%   3
%   Mutate
%   Fitness a: 1.0
%   Fitness b: 1.0
%   Chromosome terbaik: 
%   [[10, 2, 3, 1], [9, 26, 4, 1], [17, 9, 5, 1], [11, 5, 2, 3], [7, 18, 3, 3], [6, 12, 5, 3], [10, 25, 3, 4], [15, 32, 4, 4], [12, 28, 5, 4], [2, 35, 2, 5], [3, 16, 4, 5], [9, 10, 6, 5], [9, 6, 7, 6], [14, 1, 1, 6], [17, 14, 3, 6], [2, 4, 2, 6], [10, 27, 4, 6], [13, 37, 1, 7], [4, 33, 7, 7], [5, 22, 5, 7], [1, 11, 4, 7], [11, 21, 2, 8], [7, 13, 4, 8], [3, 30, 7, 8], [2, 24, 5, 9], [9, 29, 4, 10], [17, 20, 4, 11], [5, 7, 6, 11], [16, 17, 3, 12], [11, 38, 6, 12], [4, 3, 5, 12], [2, 8, 4, 12], [6, 15, 4, 13], [11, 34, 2, 14], [3, 36, 4, 14], [14, 31, 7, 14], [13, 19, 6, 14], [2, 23, 3, 14]]
% \end{lstlisting}
\begin{longtable}[c]{|c|c|c|>{\centering\arraybackslash}m{3cm}|>{\centering\arraybackslash}m{3cm}|}
  \caption{Hasil Penjadwalan Otomatis}
  \label{tab:jadwal}\\
  \hline
  Hari   & Waktu       & Ruang                  & Matkul                                                              & Dosen                                         \\ \hline
  \rowcolor[HTML]{DAE8FC} 
  Senin  & 07.30-10.00 & A235, Kapasitas: 40    & Aljabar Linear A, \linebreak Peserta: 5                             & Muhtadin, S.T., M.T.                          \\ \hline
  \rowcolor[HTML]{DAE8FC} 
  Senin  & 07.30-10.00 & A108, Kapasitas: 80    & Rangkaian Digital A, \linebreak Peserta: 24                         & Arief Kurniawan, S.T., M.T.                   \\ \hline
  \rowcolor[HTML]{DAE8FC} 
  Senin  & 07.30-10.00 & Pasca 2, Kapasitas: 40 & Desain Dan Rekayasa Sistem B, \linebreak Peserta: 40                & Dion Hayu Fandiantoro, S.T., M.Eng.           \\ \hline
  \rowcolor[HTML]{DAE8FC} 
  Senin  & 13.30-16.00 & A234, Kapasitas: 36    & Deep Learning untuk Multimedia A, \linebreak Peserta: 30            & Atar Fuady Babgei, S.T., M.Sc.                \\ \hline
  \rowcolor[HTML]{DAE8FC} 
  Senin  & 13.30-16.00 & A235, Kapasitas: 40    & Pemrograman Lanjut B, \linebreak Peserta: 39                        & Dr. Surya Sumpeno, S.T.,M.Sc.                 \\ \hline
  \rowcolor[HTML]{DAE8FC} 
  Senin  & 13.30-16.00 & Pasca 2, Kapasitas: 40 & Matematika Diskrit A, \linebreak Peserta: 32                        & Prof. Dr. Ir. Mauridhi Hery Purnomo, M.Eng.   \\ \hline
  Selasa & 07.30-10.00 & A235, Kapasitas: 40    & Probabilitas dan Statistik A, \linebreak Peserta: 28                & Muhtadin, S.T., M.T.                          \\ \hline
  Selasa & 07.30-10.00 & A108, Kapasitas: 80    & Sistem Mikroprosesor dan Mikrokontroller A, \linebreak Peserta: 26  & Ahmad Zaini, S.T., M.Sc.                      \\ \hline
  Selasa & 07.30-10.00 & Pasca 2, Kapasitas: 40 & Rangkaian Listrik B, \linebreak Peserta: 35                         & Mochamad Hariadi, S.T.,M.Sc.,Ph.D             \\ \hline
  Selasa & 10.00-12.30 & A234, Kapasitas: 36    & Sistem Operasi B, \linebreak Peserta: 26                            & Dr. Diah Puspito Wulandari, S.T.,M.Sc.        \\ \hline
  Selasa & 10.00-12.30 & A108, Kapasitas: 80    & Metode Numerik C, \linebreak Peserta: 59                            & Dr. I Ketut Eddy Purnama, S.T.,M.T.           \\ \hline
  Selasa & 10.00-12.30 & AJ401, Kapasitas: 25   & Jaringan Komputer A, \linebreak Peserta: 41                         & Arief Kurniawan, S.T., M.T.                   \\ \hline
  Selasa & 13.30-16.00 & AJ402, Kapasitas: 25   & Desain Pemrograman Game A, \linebreak Peserta: 9                    & Arief Kurniawan, S.T., M.T.                   \\ \hline
  Selasa & 13.30-16.00 & B211, Kapasitas: 25    & Dasar Pemrograman P, \linebreak Peserta: 14                         & Susi Juniastuti, S.T.,M.Eng.                  \\ \hline
  Selasa & 13.30-16.00 & A235, Kapasitas: 40    & Metode Numerik A, \linebreak Peserta: 40                            & Dion Hayu Fandiantoro, S.T., M.Eng.           \\ \hline
  Selasa & 13.30-16.00 & A234, Kapasitas: 36    & Arsitektur dan Organisasi Sistem Komputer B, \linebreak Peserta: 27 & Dr. Diah Puspito Wulandari, S.T.,M.Sc.        \\ \hline
  Selasa & 13.30-16.00 & A108, Kapasitas: 80    & Rangkaian Listrik A, \linebreak Peserta: 40                         & Muhtadin, S.T., M.T.                          \\ \hline
  \rowcolor[HTML]{DAE8FC} 
  Rabu   & 07.30-10.00 & B211, Kapasitas: 25    & Sistem Tertanam B, \linebreak Peserta: 24                           & Dr. Supeno Mardi Susiki Nugroho, S.T., M.T.   \\ \hline
  \rowcolor[HTML]{DAE8FC} 
  Rabu   & 07.30-10.00 & AJ402, Kapasitas: 25   & Sistem Mikroprosesor dan Mikrokontroller B, \linebreak Peserta: 20  & Ir. Hany Boedinugroho, M.T.                   \\ \hline
  \rowcolor[HTML]{DAE8FC} 
  Rabu   & 07.30-10.00 & Pasca 2, Kapasitas: 40 & Pengolahan Sinyal Digital B, \linebreak Peserta: 26                 & Prof. Dr. Ir. Yoyon Kusnendar Suprapto, M.Sc. \\ \hline
  \rowcolor[HTML]{DAE8FC} 
  Rabu   & 07.30-10.00 & A108, Kapasitas: 80    & Jaringan Komputer B, \linebreak Peserta: 40                         & Dr. Eko Mulyanto Yuniarno, S.T.,M.T.          \\ \hline
  \rowcolor[HTML]{DAE8FC} 
  Rabu   & 10.00-12.30 & A234, Kapasitas: 36    & Pengolahan Sinyal Digital A, \linebreak Peserta: 26                 & Atar Fuady Babgei, S.T., M.Sc.                \\ \hline
  \rowcolor[HTML]{DAE8FC} 
  Rabu   & 10.00-12.30 & A108, Kapasitas: 80    & Matematika Diskrit B, \linebreak Peserta: 59                        & Dr. Surya Sumpeno, S.T.,M.Sc.                 \\ \hline
  \rowcolor[HTML]{DAE8FC} 
  Rabu   & 10.00-12.30 & AJ402, Kapasitas: 25   & Sistem Manajemen Basis Data A, \linebreak Peserta: 25               & Dr. I Ketut Eddy Purnama, S.T.,M.T.           \\ \hline
  \rowcolor[HTML]{DAE8FC} 
  Rabu   & 13.30-16.00 & Pasca 2, Kapasitas: 40 & Persepsi Robot A, \linebreak Peserta: 13                            & Dr. Diah Puspito Wulandari, S.T.,M.Sc.        \\ \hline
  Kamis  & 07.30-10.00 & A108, Kapasitas: 80    & Sekuriti Sistem Komputer A, \linebreak Peserta: 33                  & Arief Kurniawan, S.T., M.T.                   \\ \hline
  Kamis  & 10.00-12.30 & A108, Kapasitas: 80    & Pengantar Robotika A, \linebreak Peserta: 33                        & Dion Hayu Fandiantoro, S.T., M.Eng.           \\ \hline
  Kamis  & 10.00-12.30 & AJ401, Kapasitas: 25   & Desain Pemrograman Game P, \linebreak Peserta: 8                    & Prof. Dr. Ir. Yoyon Kusnendar Suprapto, M.Sc. \\ \hline
  Kamis  & 13.30-16.00 & A235, Kapasitas: 40    & Pemrograman Lanjut A, \linebreak Peserta: 39                        & Reza Fuad Rachmadi, S.T.,M.T.,Ph.D            \\ \hline
  Kamis  & 13.30-16.00 & AJ401, Kapasitas: 25   & Visi Komputer A, \linebreak Peserta: 16                             & Atar Fuady Babgei, S.T., M.Sc.                \\ \hline
  Kamis  & 13.30-16.00 & Pasca 2, Kapasitas: 40 & Arsitektur dan Organisasi Sistem Komputer A, \linebreak Peserta: 30 & Ir. Hany Boedinugroho, M.T.                   \\ \hline
  Kamis  & 13.30-16.00 & A108, Kapasitas: 80    & Desain Dan Rekayasa Sistem A, \linebreak Peserta: 49                & Dr. Diah Puspito Wulandari, S.T.,M.Sc.        \\ \hline
  \rowcolor[HTML]{DAE8FC} 
  Jumat  & 07.30-10.00 & A108, Kapasitas: 80    & Metode Numerik B, \linebreak Peserta: 40                            & Prof. Dr. Ir. Mauridhi Hery Purnomo, M.Eng.   \\ \hline
  \rowcolor[HTML]{DAE8FC} 
  Jumat  & 13.30-16.00 & A234, Kapasitas: 36    & Sistem Operasi A, \linebreak Peserta: 25                            & Atar Fuady Babgei, S.T., M.Sc.                \\ \hline
  \rowcolor[HTML]{DAE8FC} 
  Jumat  & 13.30-16.00 & A108, Kapasitas: 80    & Sistem Tertanam A, \linebreak Peserta: 27                           & Dr. I Ketut Eddy Purnama, S.T.,M.T.           \\ \hline
  \rowcolor[HTML]{DAE8FC} 
  Jumat  & 13.30-16.00 & AJ402, Kapasitas: 25   & Sistem Manajemen Basis Data B, \linebreak Peserta: 25               & Susi Juniastuti, S.T.,M.Eng.                  \\ \hline
  \rowcolor[HTML]{DAE8FC} 
  Jumat  & 13.30-16.00 & AJ401, Kapasitas: 25   & Pemrograman Sistem dan Jaringan A, \linebreak Peserta: 11           & Dr. Supeno Mardi Susiki Nugroho, S.T., M.T.   \\ \hline
  \rowcolor[HTML]{DAE8FC} 
  Jumat  & 13.30-16.00 & A235, Kapasitas: 40    & Persamaan Differensial dan Deret A, \linebreak Peserta: 35          & Dr. Diah Puspito Wulandari, S.T.,M.Sc.        \\ \hline
\end{longtable}
Hasil keluaran program diatas merupakan hasil keluaran dari percobaan penjadwalan otomatis yang memperhatikan kapasitas ruangan dan jumlah peserta tiap mata kuliah dan menggunakan probabilitasan percobaan yang memperhatikan probabilitasis mutasi sebesar 0.05. Percobaan ini memakan waktu 14 menit 52 detik.

Terdapat dua percobaan yang dilakukan dalam sistem penjadwalan otomatis ini. Percobaan pertama yaitu tanpa memperhatikan kapasitas kelas dan jumlah perserta tiap mata kuliah, dan percobaan kedua dengan memperhatikan kapasitas ruang kelas dan jumlah perserta tiap mata kuliah.
Masing-masing percobaan dilakukan dengan memvariasikan probabilitas mutasi yaitu dengan nilai 0.04, 0.05, dan 0.06.
Masing-masing percobaan dilakukan dengan 5 kali \emph{loop} pada konfigurasi yang sama. 
Hasil percobaan yang telah dilakukan dapat dilihat pada tabel dibawah ini:
\begin{longtable}{|c|c|}
  \caption{Percobaan tanpa memperhatikan kapasitas \linebreak ruang dan jumlah peserta mata kuliah}
  \label{tab:tanpaKapasitas}\\
  \hline
  \rowcolor[HTML]{C0C0C0} 
{\color[HTML]{000000} \textbf{Probabilitas Mutasi}} & {\color[HTML]{000000} \textbf{Rata-Rata Waktu}} \\ \hline
0.04                                                & 6m 46s                                           \\ \hline
0.05                                                & 5m 38s                                           \\ \hline
0.06                                                & 4m 54s                                           \\ \hline
\end{longtable}

\begin{longtable}{|c|c|}
  \caption{Percobaan dengan memperhatikan kapasitas \linebreak ruang dan jumlah peserta mata kuliah}
  \label{tab:denganKapasitas}\\
  \hline
  \rowcolor[HTML]{C0C0C0} 
{\color[HTML]{000000} \textbf{Probabilitas Mutasi}} & {\color[HTML]{000000} \textbf{Rata-Rata Waktu}} \\ \hline
0.04                                                & 39m 43s                                         \\ \hline
0.05                                                & 14m 53s                                         \\ \hline
0.06                                                & 23m 18s                                         \\ \hline
\end{longtable}


% Pengujian dilakukan dengan memvariasikan nilai probabilitas mutasi yang kemudian diperoleh rataan waktu untuk menyelesaikan proses penjadwalan. 
% Tabel rata-rata waktu proses penjadwalan adalah sebagai berikut.
% \begin{longtable}{|c|c|}
%   \caption{Hasil Percobaan Variasi Probabilitas Mutasi}
%   \label{tab:variasiMutasi}\\
%   \hline
%   \rowcolor[HTML]{C0C0C0} 
%   {\color[HTML]{000000} \textbf{Probabilitas Mutasi}} & {\color[HTML]{000000} \textbf{Rata-Rata Waktu}} \\ \hline
%   0.04                                                & 39m 43s                                         \\ \hline
%   0.05                                                & 14m 53s                                         \\ \hline
%   0.06                                                & 23m 18s                                         \\ \hline
% \end{longtable}
\section{Pembahasan}
\label{sec:Pembahasan}

Dari pengujian yang telah dilakukan dapat diketahui bahwa semakin banyak batasan-batasan yang diberikan dalam proses algoritma genetika, maka kompleksitas dari proses perhitungan evaluasi fitness juga akan semakin rumit. 
Kerumitan proses perhitungan ini tentu akan menambah durasi dari proses algoritma genetika itu sendiri. Hal ini dapat dilihat dari perbedaan waktu antara Tabel \ref{tab:tanpaKapasitas} dengan Tabel \ref{tab:denganKapasitas}.

Selain kompleksitas perhitungan evaluasi fitness dari satu individu, penentuan probabilitas mutasi juga sangat berpengaruh terhadap durasi proses algoritma genetika. Probabilitas \linebreak mutasi dalam sebuah proses algoritma genetika tidak boleh terlalu besar, karena mengakibatkan hilangnya kemiripan antara individu baru dengan individu induknya. 
Sedangkan jika probabilitas mutasi bernilai terlalu kecil, maka durasi untuk memperoleh solusi berupa individu terbaik, akan memakan waktu yang sangat lama dan membuat proses algorima genetika dari penjadwalan otomatis ini menjadi tidak optimal. 

Data yang diperoleh pada Tabel \ref{tab:tanpaKapasitas} menunjukkan bahwa probabilitas mutasi 0.06 memiliki rata-rata durasi yang paling singkat yaitu dengan waktu 4 menit 54 detik. Sedangkan data pada Tabel \ref{tab:denganKapasitas} menunjukkan bahwa probabilitas mutasi 0.05 memiliki rata-rata durasi yang paling singkat yakni dengan waktu 14 menit 53 detik. 
% Contoh pembuatan tabel
% \begin{longtable}{|c|c|c|}
%   \caption{Hasil Pengukuran Energi dan Kecepatan}
%   \label{tb:EnergiKecepatan}                                   \\
%   \hline
%   \rowcolor[HTML]{C0C0C0}
%   \textbf{Energi} & \textbf{Jarak Tempuh} & \textbf{Kecepatan} \\
%   \hline
%   10 J            & 1000 M                & 200 M/s            \\
%   20 J            & 2000 M                & 400 M/s            \\
%   30 J            & 4000 M                & 800 M/s            \\
%   40 J            & 8000 M                & 1600 M/s           \\
%   \hline
% \end{longtable}

% \lipsum[2-4]

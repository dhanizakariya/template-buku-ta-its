\chapter{PENUTUP}
\label{chap:penutup}

% Ubah bagian-bagian berikut dengan isi dari penutup

\section{Kesimpulan}
\label{sec:kesimpulan}

Berdasarkan hasil pengujian yang telah dilakukan, didapatkan beberapa kesimpulan \linebreak sebagai berikut:

\begin{enumerate}[nolistsep]

  \item Program yang dibuat mampu menghasilkan jadwal untuk satu pekan perkuliahan.
  \item Nilai dari probabilitas mutasi akan mempengaruhi durasi dari penyelesaian proses \linebreak penjadwalan. 
        Semakin besar nilai probabilitas mutasi maka penyelesaian proses penjadwalan akan semakin lama, dan semakin mendekati 0 nilai dari probabilitas mutasi maka proses penyelesaian proses penyelesaian proses penjadwalan juga semakin lama.
  \item Jumlah constraint pada perhitungan nilai fitness mempengaruhi durasi penyelesaian \linebreak proses penjadwalan. 
        Semakin banyak constraint, maka proses penjadwalaan akan \linebreak semakin lama.

\end{enumerate}

\section{Saran}
\label{chap:saran}

Untuk pengembangan lebih lanjut pada penelitian ini, terdapat beberapa saran antara lain:

\begin{enumerate}[nolistsep]
  \item Pada program ini semua mata kuliah dianggap memiliki 3 sks. Oleh karena itu pada penelitian lebih lanjut untuk memperhatikan bobot dari masing-masing mata kuliah.
  \item Menambahkan fitur \emph{output} jadwal secara otomatis dalam bentuk tabel excel sehingga tidak perlu melakukannya secara manual.
  \item Menambahkan fitur input data untuk meningkatkan fleksibilitas program sehingga bisa menyesuaikan dengan kondisi yang dibutuhkan.
  \item Menambahkan GUI dan membuat program menjadi \emph{executable program} sehingga \linebreak mempermudah \emph{user} dalam menggunakan program penjadwalan otomatis ini tanpa harus menggunakan aplikasi lain.
\end{enumerate}

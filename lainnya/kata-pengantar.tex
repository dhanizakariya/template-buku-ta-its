\begin{center}
  \Large
  \textbf{KATA PENGANTAR}
\end{center}

\addcontentsline{toc}{chapter}{KATA PENGANTAR}

\vspace{2ex}

% Ubah paragraf-paragraf berikut dengan isi dari kata pengantar

Puji dan syukur kehadirat Allah Swt. atas segala limpahan berkah, rahmat, dan hidayah-Nya, penulis dapat menyelesaikan penelitian
ini dengan judul \emph{\textbf{Penjadwalan Perkuliahan di Departemen Teknik Komputer ITS Berbasis Algoritma Genetika.}}

Penelitian ini disusun dalam rangka pemenuhan bidang riset di
Departmen Teknik Komputer, serta digunakan sebagai persyaratan
menyelesaikan pendidikan S1. Penelitian ini dapat terselesaikan
tidak lepas dari bantuan sebagai pihak.
Oleh karena itu, penulis mengucapkan terima kasih kepada:

\begin{enumerate}[nolistsep]

  \item Keluarga, Ibu, Bapak, dan Adik - Adik tercinta yang telah
  memberikan dorongan spiritual dan material dalam penyelesaian buku penelitian ini.

  \item Bapak Dr. Supeno Mardi Susiki Nugroho, ST., MT. selaku Kepala Departemen Teknik Komputer, Fakultas Teknologi
  Elektro dan Informatika Cerdas (FTEIC), Institut Teknologi
  Sepuluh Nopoember.

  \item Bapak Dr. Diah Puspito Wulandari, S.T., M.Sc. selaku dosen pembimbing I yang selalu memberikan arahan dan saran
  selama mengerjakan penelitian tugas akhir ini.

  \item Bapak Dr. Supeno Mardi Susiki Nugroho, ST., MT. selaku dosen pembimbing II yang selalu memberikan arahan dan
  saran selama mengerjakan penelitian tugas akhir ini.

  \item Bapak-ibu dosen pengajar Departemen Teknik Komputer atas
  pengajaran, bimbingan, serta perhatian yang diberikan kepada penulis selama ini.

  \item Seluruh teman-teman Teknik Komputer ITS, khususnya teman-teman Teknik Komputer angkatan 2019. 

\end{enumerate}

Akhir kata, kesempurnaan hanya milik Allah SWT, untuk itu penulis memohon segenap kritik dan saran yang membangun. Semoga penelitian ini dapat memberikan manfaat bagi kita semua, aamiin.

\begin{flushright}
  \begin{tabular}[b]{c}
    \place{}, \MONTH{} \the\year{} \\
    \\
    \\
    \\
    \\
    \name{}
  \end{tabular}
\end{flushright}

% Atur variabel berikut sesuai namanya

% nama
\newcommand{\name}{Muhammad Zakariya Nur Ramdhani}
\newcommand{\authorname}{Ramdhani, Muhammad Zakariya}
\newcommand{\nickname}{Ramdhani}
\newcommand{\advisor}{Dr.\ Diah Puspito Wulandari, S.T., M.Sc}
\newcommand{\coadvisor}{Dr.\ Supeno Mardi Susiki Nugroho, ST., MT.}
\newcommand{\examinerone}{Dr. Galileo Galilei, S.T., M.Sc}
\newcommand{\examinertwo}{Friedrich Nietzsche, S.T., M.Sc}
\newcommand{\examinerthree}{Alan Turing, ST., MT}
\newcommand{\headofdepartment}{Dr.\ Supeno Mardi Susiki Nugroho, ST., MT.}

% identitas
\newcommand{\nrp}{0721 19 4000 0016}
\newcommand{\advisornip}{19801219 200501 2 000}
\newcommand{\coadvisornip}{19700313 199512 1 001}
\newcommand{\examineronenip}{18560710 194301 1 001}
\newcommand{\examinertwonip}{18560710 194301 1 001}
\newcommand{\examinerthreenip}{18560710 194301 1 001}
\newcommand{\headofdepartmentnip}{19700313 199512 1 001}

% judul
\newcommand{\tatitle}{PEMANFAATAN ALGORITMA GENETIKA UNTUK PROSES PENJADWALAN PERKULIAHAN DI DEPARTEMEN TEKNIK KOMPUTER ITS}
\newcommand{\engtatitle}{\emph{UTILIZATION OF GENETIC ALGORITHM FOR LECTURE SCHEDULING PROCESS IN COMPUTER ENGINEERING DEPARTMENT ITS}}

% tempat
\newcommand{\place}{Surabaya}

% jurusan
\newcommand{\studyprogram}{Teknik Komputer}
\newcommand{\engstudyprogram}{Computer Engineering}

% fakultas
\newcommand{\faculty}{Teknologi Elektro dan Informatika Cerdas}
\newcommand{\engfaculty}{Intelligent Electrical and Informatics Technology}

% singkatan fakultas
\newcommand{\facultyshort}{FTEIC}
\newcommand{\engfacultyshort}{ELECTICS}

% departemen
\newcommand{\department}{Teknik Komputer}
\newcommand{\engdepartment}{Computer Engineering}

% kode mata kuliah
\newcommand{\coursecode}{EC224801}
